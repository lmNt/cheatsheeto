\begin{center}
     \Large{\textbf{Numerik Cheat Sheeto}} \\
\end{center}

\section{Basics}
\subsection{Sortieren}
\subsection{FFT}

\section{Lineare Gleichungssysteme}
\subsection{Allgemeine Aufgabenstellung}
Geg.: $\v{A} \in \Rr{n}{n}$, $\v{b} \in \R{n}$ \\
Ges.: $\v{x} \in \R{n}$ \\
\centering $\v{Ax}=\v{b}$ \flushleft
\subsection{Dreiecksmatrizen}
Untere Dreiecksmatrix $\v{L} \in \Rr{n}{n}$ und obere Dreiecksmatrix $\v{R} \in \Rr{n}{n}$.\\
\textsc{Regul\a re/invertierbare/nicht-singul\a re Matrix}\\
Matrix $\v{A}$ ist regul\a r, wenn $\det \v{A}\neq 0$. Determinante einer $\Delta$Matrix ist das Produkt ihrer Diagonalelemente. $\v{L}$ und $\v{R}$ sind regul\a r, wenn alle Diagonalelemente $\neq 0$.\\

\textsc{Vorw\a rtseinsetzen}\\
\centering $\v{Ly}=\v{b}$ \flushleft
Rechenaufwand: $n^2$ AO \\
Um Speicher zu sparen  $b_i \leftarrow y_i$.\vspace{0.1cm}

{\addtolength{\leftskip}{20mm}
\hrulefill\\
for $j=1:n$\\
\quad $x_j \leftarrow b_j/l_{jj}$\\
\quad for $i=j+1:n$\\
\qquad $b_i \leftarrow b_i-l_{ij}x_j$\\
\hrulefill\\
}

\textsc{R\u ckw\a rtseinsetzen}\\
\centering $\v{Rx}=\v{y}$ \flushleft 
Rechenaufwand: $n^2$ AO \\
Um Speicher zu sparen  $b_i \leftarrow x_i$.\vspace{0.1cm}

{\addtolength{\leftskip}{20mm}
\hrulefill\\
for $j=n:1$\\
\quad $x_j \leftarrow b_j/r_{jj}$\\
\quad for $i=1:j-1$\\
\qquad $b_i \leftarrow b_i-r_{ij}x_j$\\
\hrulefill\\
}

\subsection{LR-Zerlegung}
Sei $\v{A} \in \Rr{n}{n}$ und $\v{L},\v{R} \in \Rr{n}{n}$\\
\centering $\v{A}=\v{LR}$ \flushleft

Ansatz: \\
Matrizen $\v{A}$, $\v{L}$, $\v{R}$ in Teilmatrizen $\v{A}_{**}$, $\v{A}_{*1}$, $\v{A}_{1*}$, $\v{L}_{**}$, $\v{L}_{*1}$, $\v{R}_{**}$, $\v{R}_{1*}$ zerlegen.\\ Es folgen 4 Gleichungen aus $\v{A}=\v{LR}$:
\begin{align*}
a_{11}     &= l_{11}r_{11} & & \\
\v{A}_{*1} &= \v{L}_{*1}r_{11} &\Leftrightarrow \v{L}_{*1} &= \v{A}_{*1}/r_{11} \\ 
\v{A}_{1*} &= l_{11}\v{R}_{1*} &\Leftrightarrow \v{R}_{1*} &= \v{A}_{1*}\\
\v{A}_{**} &= \v{L}_{*1}\v{R}_{1*}+\v{L}_{**}\v{R}_{**} & &
\end{align*}
Per Def. $l_{11}=1$ und damit $r_{11}=a_{11}$, sodass
\begin{equation}
\v{A}_{**} - \v{L}_{*1}\v{R}_{1*} = \v{L}_{**}\v{R}_{**}
\end{equation}

\textsc{Praktische Umsetzung}\\
Elemente von $\v{A}$ \u berschreiben, sodass:
\begin{equation}
\begin{pmatrix}
r_{11} & r_{12} & \cdots    & r_{1n} \\
l_{21} & r_{22} & \ddots    & \vdots \\
\vdots & \ddots & \ddots    & r_{n-1,n} \\
l_{n1} & \cdots & l_{n,n-1} & r_{nn} 
\end{pmatrix}
\end{equation}


\subsection{Fehlerverst\a rkung}
\subsection{QR-Zerlegung}
\subsection{Ausgleichsprobleme}


\section{Nichtlineare Gleichungssysteme}
\subsection{Bisektionsverfahren}
\subsection{Allgemeine Fixpunktiterationen}
\subsection{Newton-Verfahren}

\section{Eigenwertprobleme}
\subsection{Vektoriteration}
\subsection{Inverse Iteration}
\subsection{Orthogonale Iteration}
\subsection{QR-Iteration}
\subsection{Praktische QR-Iteration}

\section{Approximation von Funktionen}
\subsection{Polynominterpolation}
\subsection{Neville-Aitken-Verfahren}
\subsection{Newtons dividierte Differenzen}
\subsection{Approximation von Funktionen}

\section{Numerische Integration}
\subsection{Quadraturformeln}
\subsection{Fehleranalyse}
\clearpage