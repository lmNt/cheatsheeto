\begin{center}
     \Large{\textbf{Numerik Cheat Sheeto}} \\
\end{center}

\section{Basics}
\subsection{Sortieren}
\subsection{FFT}

\section{Lineare Gleichungssysteme}
\subsection{Allgemeine Aufgabenstellung}
Geg.: $\v{A} \in \Rr{n}{n}$, $\v{b} \in \R{n}$ \\
Ges.: $\v{x} \in \R{n}$ \\
\centering $\v{Ax}=\v{b}$ \flushleft
\subsection{Dreiecksmatrizen}
Untere Dreiecksmatrix $\v{L} \in \Rr{n}{n}$ und obere Dreiecksmatrix $\v{R} \in \Rr{n}{n}$.\vspace{0.2cm}

\textsc{Regul\a re/invertierbare/nicht-singul\a re Matrix}\\
Matrix $\v{A}$ ist regul\a r, wenn $\det \v{A}\neq 0$. Determinante einer $\Delta$Matrix ist das Produkt ihrer Diagonalelemente. $\v{L}$ und $\v{R}$ sind regul\a r, wenn alle Diagonalelemente $\neq 0$.\vspace{0.2cm}

\textsc{Vorw\a rtseinsetzen}\\
\centering $\v{Ly}=\v{b}$ \flushleft
Rechenaufwand: $n^2$ AO \\
Um Speicher zu sparen  $b_i \leftarrow y_i$.\vspace{0.2cm}

\verb!forward_subst!\\
{\addtolength{\leftskip}{0mm}
\hrulefill\\
for $j=1:n$\\
\quad $x_j \leftarrow b_j/l_{jj}$\\
\quad for $i=j+1:n$\\
\qquad $b_i \leftarrow b_i-l_{ij}x_j$\\
\hrulefill\\
}\vspace{0.2cm}

\textsc{R\u ckw\a rtseinsetzen}\\
\centering $\v{Rx}=\v{y}$ \flushleft 
Rechenaufwand: $n^2$ AO \\
Um Speicher zu sparen  $b_i \leftarrow x_i$.\vspace{0.2cm}

\verb!backward_subst!\\
{\addtolength{\leftskip}{0mm}
\hrulefill\\
for $j=n:1$\\
\quad $x_j \leftarrow b_j/r_{jj}$\\
\quad for $i=1:j-1$\\
\qquad $b_i \leftarrow b_i-r_{ij}x_j$\\
\hrulefill\\
}\vspace{0.2cm}

\subsection{LR-Zerlegung}
Sei $\v{A} \in \Rr{n}{n}$ und $\v{L},\v{R} \in \Rr{n}{n}$\\
\centering $\v{A}=\v{LR}$ \flushleft

Ansatz: \\
Matrizen $\v{A}$, $\v{L}$, $\v{R}$ in Teilmatrizen $\v{A}_{**}$, $\v{A}_{*1}$, $\v{A}_{1*}$, $\v{L}_{**}$, $\v{L}_{*1}$, $\v{R}_{**}$, $\v{R}_{1*}$ zerlegen.\\ Es folgen 4 Gleichungen aus $\v{A}=\v{LR}$:
\begin{align*}
a_{11}     &= l_{11}r_{11} & & \\
\v{A}_{*1} &= \v{L}_{*1}r_{11} &\Leftrightarrow \v{L}_{*1} &= \v{A}_{*1}/r_{11} \\ 
\v{A}_{1*} &= l_{11}\v{R}_{1*} &\Leftrightarrow \v{R}_{1*} &= \v{A}_{1*}\\
\v{A}_{**} &= \v{L}_{*1}\v{R}_{1*}+\v{L}_{**}\v{R}_{**} & &
\end{align*}
Per Def. $l_{11}=1$ und damit $r_{11}=a_{11}$, sodass
\begin{equation}
\v{A}_{**} - \v{L}_{*1}\v{R}_{1*} = \v{L}_{**}\v{R}_{**}
\end{equation}

\textsc{Praktische Umsetzung}\\
Elemente von $\v{A}$ \u berschreiben, sodass:
\begin{equation}
\begin{pmatrix}
r_{11} & r_{12} & \cdots    & r_{1n} \\
l_{21} & r_{22} & \ddots    & \vdots \\
\vdots & \ddots & \ddots    & r_{n-1,n} \\
l_{n1} & \cdots & l_{n,n-1} & r_{nn} 
\end{pmatrix}
\end{equation}

\textsc{Kriterium}\\
Sei $\v{A}$ regul\a r, $\v{A}$ besitzt eine LR-Zerlegung $\Leftrightarrow$ Alle Hauptuntermatrizen regul\a r.\vspace{0.2cm}

\textsc{Modellproblem: Bandmatrix}\\
Irgendwas bzgl Effizienz.\vspace{0.2cm}

\textsc{LR-Decomp}\\
Aufwand: \emph{kubisch} \\
$\downarrow$ Aufwand: Nur 1x f\u r jede Matrix betreiben. Sobald LR-Decomp vorliegt nur noch \emph{quadratischer} Aufwand \\ 
$\Downarrow$ Aufwand: Tridiagonalmatrix. Erster Schritt mit 3 AOPs. Restmatrix bleibt tridiagonal. Aufwand $3n$ + $6n$ f\u r R-und F-Einsetzen.\vspace{0.2cm}

\verb!lr_decomp!\\
{\addtolength{\leftskip}{0mm}
\hrulefill\\
for $k=1:n$\\
\quad for $i=k+1:n$ \\ 
\qquad $a_{ik} \leftarrow a_{ik}/a_{kk}$ \\
\qquad for $j=k+1:n$\\
\qquad\quad $a_{ij} \leftarrow a_{ij}-a_{ik}a_{kj}$\\
\hrulefill\\
}\vspace{0.2cm}

\textsc{Problem der Existenz einer LR-Z}\\
Falls $\v{A}$ oder $\v{A}_{**}$ eine 0 auf der Diagonalen hat, existiert keine LR-Zerlegung. L\o sung: Permutiere die Zeilen von $\v{A}$ so, dass das Ergebnis eine LR-Z besitzt.\vspace{0.2cm}

\textsc{Permutationsmatrix}\\
Sei $\v{P} \in \Rr{n}{n}$. Falls in jeder Zeile und Spalte von $\v{P}$ genau ein Eintrag 1 und alle anderen 0, dann ist $\v{P}$ eine Permutationsmatrix. $\v{P}$ ist orthogonal. Ein Produkt zweier Permutationsmatrizen $\v{P}\v{Q}$ ist auch eine Permutationsmatrix.\vspace{0.2cm}

\textsc{Permutation}\\
Bijektive Abbildung $\pi : \{1,\cdots,n\} \rightarrow \{1,\cdots,n\}$.\vspace{0.2cm}

\textsc{LR-Z mit Pivotsuche}\\
$\v{A}$ regul\a r. Es existiert $\v{P} \in \Rr{n}{n}$ sodass $\v{P}\v{A}=\v{L}\v{R}$ gilt. \\
Pivotisierung: Finde betragsmaximalstes Element in der aktuellen Spalte, welches unter dem aktuellen Diagonalelement von $\v{A}$ liegt und tausche die aktuelle Zeile mit der Zeile in der das betragsmaximalste Element ist, mit Hilfe von $\v{P}$. $a_{11} \neq 0$, da betragsgr\o \ss tes Element.\vspace{0.2cm}

\textsc{L\o sen eines Gleichungssystems mit Pivotsuche}\\
$\v{A}\v{x} = \v{b} \Leftrightarrow \v{P}\v{A}\v{x} = \v{P}\v{b} \Leftrightarrow \v{L}\v{R}\v{x}=\v{P}\v{b}$.\\
1.) $\v{Ly}=\tilde{\v{b}}$ \quad 2.) $\v{Rx} = \v{y}$

\verb!lr_pivot!\\
{\addtolength{\leftskip}{0mm}
\hrulefill\\
for $k=1:n$\\
\quad $i_* \leftarrow k$ \qquad\qquad // Finde max. Element \\ 
\quad for $i=k+1:n$ \\ 
\qquad if $|a_{ik}| > |a_{i_*k}|$: $i_* \leftarrow i$ \\
\quad $p_k \leftarrow i_*$ \\
\quad for $j=1:n$ \qquad // Tausche Zeilen \\ 
\qquad $\gamma \leftarrow a_{kj}$, $a_{kj} \leftarrow a_{i_*j}$, $a_{i_*j} \leftarrow \gamma$\\
\quad for $i=k+1:n$ \\ 
\qquad $a_{ik} \leftarrow a_{ik}/a_{kk}$ \\
\qquad for $j=k+1:n$\\
\qquad\quad $a_{ij} \leftarrow a_{ij}-a_{ik}a_{kj}$\\
\hrulefill\\
}
$\v{p}$ protokolliert, welche Vertauschungen durchgef\u hrt wurden, um sie sp\a ter auf $\v{b}$ anwenden zu k\o nnen.\\
Aufwand: $\frac{2}{3}n^3$.\vspace{0.2cm}

\textsc{Sonderfall: $\v{A}$ positiv definit}\\
TODO.


\subsection{Fehlerverst\a rkung}
\textsc{Norm des Matrix-Vektor-Produkts}\\
\emph{Wie stark \a ndert sich die L\a nge eines Vektors wenn er mit $\v{A}$ multipliziert wird. Mapping von Einheitskreis auf Ellipse.}.
F\u r $\v{A} \in \Rr{n}{n}$ gilt:
\begin{align*}
\alpha_2(\v{A}) &= \min\{\Vert \v{Ay} \Vert_2 : \v{y}\in\R{n}, \Vert \v{y} \Vert_2 =1 \} \\
\beta_2(\v{A}) &= \max\{\Vert \v{Ay} \Vert_2 : \v{y}\in\R{n}, \Vert \v{y} \Vert_2 =1 \} 
\end{align*}
und
\begin{align*}
\alpha_2(\v{A})\Vert \v{z} \Vert_2 \leq \Vert \v{Az} \Vert_2 \leq \beta_2(\v{A})\Vert \v{z} \Vert_2
\end{align*}
Eigenschaften der Norm:
\begin{align*}
\Vert \v{x} \Vert &= 0 \Leftrightarrow \v{x} = 0 \\
\Vert \lambda \v{x} \Vert &= | \lambda | \Vert \v{x} \Vert \\
\Vert \v{x} + \v{y} \Vert &\leq \Vert \v{x} \Vert+\Vert \v{y} \Vert
\end{align*}
TODO

\subsection{QR-Zerlegung}
\emph{F\u r jede Matrix gibt es eine QR-Z}.\\

\textsc{LR-Z: Schlecht konditioniertes Problem}\\
$\kappa_2(\v{A}) \gg 1$, $\kappa_2(\v{A}) \leq \kappa_2(\v{L})\kappa_2(\v{R})$\\
Kritisch falls $\kappa_2(\v{L})\kappa_2(\v{R}) \gg \kappa_2(\v{A})$.\\

Ziel: Suche Transformationen $\v{Q} \in \Rr{n}{n}$, die die Norm unver\a ndert lassen:
\begin{align*}
\Vert \v{Qy} \Vert_2 = \Vert y \Vert_2
\end{align*}
Mit Hinzunahme des Skalarprodukts:
\begin{align*}
\langle \v{y}, \v{y} \rangle_2 = \Vert \v{y} \Vert_2^2 = \Vert \v{Qy} \Vert_2^2 = \langle \v{Qy}, \v{Qy} \rangle_2 = \langle \v{y}, \v{Q}^*\v{Qy} \rangle_2
\end{align*}
muss $\v{Q}^*\v{Q} = \v{I}$ gelten.

Gesucht: $\v{A}=\v{QR}$.\\
Konditionszahl bzw. Fehlerverst\a rkung wird nicht verschlechtert: $\alpha_2(\v{A}) = \alpha_2(\v{R})$, $\beta_2(\v{A}) = \beta_2(\v{R})$\\
$\kappa_2(\v{A})= \kappa_2(\v{R})$.\\

\textsc{Givens-Rotation}\\
Mit Hilfe von Givens-Rotationen k\o nnen wir beliebige $\v{A}\in \Rr{m}{n}$ auf obere $\Delta$gestalt bringen.
\begin{align*}
\v{Q}=\begin{pmatrix}
c & s \\
-s & c
\end{pmatrix}
\text{ und }
\v{Qy} = \begin{pmatrix}
cy_1 + sy_2 \\ 0
\end{pmatrix}
\end{align*}
Konsekutiv Givens-Rotationen $\v{Q}_{ij}$ $i$-te und $j$-te Zeile anwenden um Eintrag $a_{ij}$ zu beseitigen. Bsp. $\v{A} \in \Rr{4}{3}$:
\begin{align*}
\v{R} = \v{Q}_{43}\v{Q}_{32}\v{Q}_{42}\v{Q}_{21}\v{Q}_{31}\v{Q}_{41}\v{A} \\
\underbrace{\v{Q}_{41}^*\v{Q}_{31}^*\v{Q}_{21}^*\v{Q}_{42}^*\v{Q}_{32}^*\v{Q}_{43}^*}_{\v{Q}}\v{R}=\v{A}
\end{align*}

\textsc{Kompakte Darstellung}\\
\emph{Verwende Nulleintr\a ge von $\v{A}$ bzw. $\v{R}$ um $\v{Q}_{ij}$ zu beschreiben}.\\
Finde Givens-Rotation:
\begin{align*}
\rho = 
\begin{cases}
s = \rho, c=\sqrt{1-s^2}  &\text{falls $|\rho|<1$} \\
c = 1/\rho, s=\sqrt{1-c^2} &\text{falls $|\rho|>1$} \\
c = 1, s=0 &\text{falls $\rho=1$}
\end{cases}
\end{align*}

Speichern der QR-Z in $\v{A}$:
\begin{equation}
\begin{pmatrix}
r_{11} & r_{12} & \cdots    & r_{1n} \\
\rho_{21} & r_{22} & \ddots    & \vdots \\
\vdots & \ddots & \ddots    & r_{n-1,n} \\
\rho_{n1} & \cdots & \rho_{n,n-1} & r_{nn} 
\end{pmatrix}
\end{equation}

\verb!Qr Decomp von! $\v{A} \in \Rr{m}{n}$\\
{\addtolength{\leftskip}{0mm}
\hrulefill\\
for $k=1:\min(m,n)$ \qquad {\scriptsize // Loop \u ber Diagonale}\\
\quad for $i=k+1:m$ \qquad {\scriptsize // Loop \u ber Elemente unter Diagonalen}\\
\qquad if $a_{ik}=0$\\
\qquad\quad $\rho \leftarrow 1$, $c \leftarrow 1$, $s \leftarrow 0$ \\
\qquad else if $|a_{kk}|\geq|a_{ik}|$ \qquad {\scriptsize // Vgl. mit Diag.element}\\
\qquad\quad $\tau \leftarrow a_{ik}/a_{kk}$, $\rho \leftarrow \tau/\sqrt{\tau^2+1}$, $s \leftarrow \rho$, $c\leftarrow \sqrt{1-s^2}$ \\
\qquad else \qquad\qquad\qquad\quad\qquad {\scriptsize // Vgl. mit Diag.element} \\
\qquad\quad $\tau \leftarrow a_{kk}/a_{ik}$, $\rho \leftarrow \sqrt{\tau^2+1}/\tau$, $c \leftarrow 1/\rho$, $s\leftarrow \sqrt{1-c^2}$ \\
\qquad {\scriptsize // Diag.element aktual., Giv.-Rot. in aktueller It. speichern}\\ 
\qquad $a_{kk} \leftarrow ca_{kk}+sa_{ik}$, $a_{ik}\leftarrow \rho$\\
\qquad for $j=k+1:n$ {\scriptsize // Loop \u ber Elemente in der $k$-ten Zeile}\\
\qquad \quad {\scriptsize // Giv-Rot auf Zeile anwenden} \\
\qquad \quad $\alpha \leftarrow a_{kj}$, $a_{kj} \leftarrow c\alpha+sa_{ij}$, $a_{ij} \leftarrow -s\alpha + ca_{ij}$ \\
\hrulefill\\
}
Aufwand: $6n^2+2n^3$ (quadratische Matrix) ~3x mehr als LR-Z. \vspace{0.2cm}

\textsc{L\o sen Gleichungssystem $\v{Ax}=\v{b}$}\\
$\v{b}=\v{Ax}=\v{QRx}=\v{Qy}$ \quad $\Leftrightarrow$ \quad 1.) $\v{y}=\v{Q}^*\v{b}$ \quad 2.) $\v{y}=\v{Rx}$.

1.) Qr\_transform: \U ber einzelne $G_{ij}$ (oben links angefangen) iterieren und auf $b$ multiplizieren. \\
2.) R\u ckw\a rtseinsetzen.\vspace{0.2cm}

\textsc{Effizientere QR-Z}\\
Householder-Spiegelungen: Aufwand ~2x mehr als LR-Z.\\
Mit Optimierungen bei Speicherzugriffen bei QR-Z \a hnlich schnell wie LR-Z.\vspace{0.2cm}

\subsection{Ausgleichsprobleme}
\emph{Wir suchen $\v{x}$ so, dass alle Gleichungen m\o glichst gleich gut erf\u llt werden}.

\section{Nichtlineare Gleichungssysteme}
\emph{Wir untersuchen nichtlineare Gleichungssysteme der Form}
\begin{align*}
&\text{Gegeben eine stetige Funktion $f:\R{n} \rightarrow \R{n}$, finde $\v{x}^* \in \R{n}$ mit} \\
&f(\v{x}^*) = \v{0}
\end{align*}
\emph{Transformieren in ein Nullstellenproblem}.
\subsection{Bisektionsverfahren}
\emph{Einfache Technik, das in jedem Schritt den Fehler mindestens halbiert. Basierend auf dem Zwischenwertsatz f\u r stetige Funktionen.}\\
\textbf{Funkioniert nur f\u r 1D.}\vspace{0.2cm}

\textsc{Zwischenwertsatz}\\
Eine reele Funktion $f$, die in $[a,b]$ stetig ist, nimmt jeden Wert zwischen $f(a)$ und $f(b)$ an. Haben $f(a)$ und $f(b)$ verschiedene Vorzeichen, so ist eine Existenz mindestens einer Nullstelle in $[a,b]$ garantiert.\vspace{0.2cm}

\textsc{Verfahren in mathematischer Notation}\\
\begin{align*}
(a^{(0)}, b^{(0)}) &= (a,b)\\
x^{(m)} &= \frac{a^{(m)} + b^{(m)}}{2} \\
(a^{(m+1)}, b^{(m+1)}) &= \begin{cases}
(a^{(m)}, x^{(m)}) & \text{ if } f(a^{(m)})f(x^{(m)}) < 0 \\
(x^{(m)}, b^{(m)}) & \text{ sonst.}
\end{cases}
\end{align*}


\verb!Bisection!\\
{\addtolength{\leftskip}{0mm}
\hrulefill\\
$f_a \leftarrow f(a)$, $f_b \leftarrow f(b)$\\
\verb!while! $b-a>\epsilon$\\
\quad $x \leftarrow (a+b)/2$ \\
\quad $f_x \leftarrow f(x)$ \\
\quad \verb!if! $f_af_x < 0$ \\ 
\qquad $b \leftarrow x$, $f_b \leftarrow f_x$ \\
\quad \verb!else! \\
\qquad $a \leftarrow x$, $f_a \leftarrow f_x$ \\
\hrulefill\\
}
Aufwand: $m+2$ Auswertungen von $f$ und $2m$ Rechenoperationen, mit $m=\lceil \log_2((b-a)/\epsilon)\rceil$ Schritten. Hohe Stabilit\a t und jedes konstruierte Intervall muss eine Nullstelle enthalten. Nur auf reelwertige Funktionen auf geeigneten Intervallen anwendbar.

\subsection{Allgemeine Fixpunktiterationen}
\textsc{Iteration}\\
$U \subseteq \R{n}$ eine abgeschlossene Teilmenge und $\Phi: U \rightarrow U$ eine (Selbst-)Abbildung. Dann ist $\Phi$ eine Iteration auf $U$. Folge der Iterierten $\v{x}^{(m+1)} = \Phi(\v{x}^{(m)})$. Kontruiere Iteration so, dass $\Phi$ gegen gesuchte L\o sung $\v{x}^*$ konvergiert. Es soll $\phi(\v{x}^*) = \v{x}^*$ gelten. \emph{Die L\o sung muss ein Fixpunkt von $\Phi$ sein.}\vspace{0.2cm}

\textsc{Mittelwertsatz der Differentialrechnung}\\
Zwischen $a$ und $b$ von $f$ gibt es mindestens einen Kurvenpunkt, f\u r den die Tangente an $\eta$ parallel zur Sekante durch $a$ und $b$ ist:
$(b-a)f'(\eta) = f(b)-f(a)$. \vspace{0.2cm}

\textsc{TODO: W\a hlen der richtigen Iteration}\\

\textsc{Fixpunktsatz von Banach}\\
Sei $\Phi$ eine Iteration auf einer \textbf{abgeschlossenen} Menge $U \subseteq \R{n}$ (\textbf{Selbstabbildung}). Sei $L \in [0,1)$ so gegeben, dass: $\Vert \Phi(\v{x}) - \Phi(\v{y}) \Vert \leq L \Vert \v{x} - \v{y} \Vert$ f\u r alle $\v{x},\v{y} \in U$. \textbf{Kontraktion}. $\Phi$ besitzt \textbf{genau} einen Fixpunkt und die Folge der Iterierten konvergiert f\u r jeden Startwert $\v{x}^{(0)} \in U$ gegen diesen Fixpunkt.\\
Fehlerabsch\a tzung \emph{a-priori} (Vorhersagen wieviele Schritte) und \emph{a-posteriori} (Pr\u fen ob N\a herung schon genau genug):
\begin{align*}
\Vert \v{x}^{(m)} - \v{x}^* \Vert &\leq \frac{L^m}{1-L} \Vert \v{x}^{(1)} - \v{x}^{(0)} \Vert \\
\Vert \v{x}^{(m)} - \v{x}^* \Vert &\leq \frac{1}{1-L} \Vert \v{x}^{(m)} - \v{x}^{(m+1)} \Vert \quad \forall m \in \mathbb{N}_0
\end{align*}

\subsection{1D-Newton-Verfahren}
$U \subseteq \R{n}$ offene Menge und $f: U \rightarrow \R{n}$ zweimal stetig differenzierbar mit Nullstelle $v{x}^*\in U$, so dass $f(\v{x}^*)=\v{0}$. Ziel: Konstruiere Iteration $\Phi$, die $\v{x}^*$ als Nullstelle besitzt.\vspace{0.2cm}

\textsc{Taylor} \\
\textbf{TODO}\\
\begin{equation*}
0 = f(x^*) = f(x) + f'(x)(x^* - x) + \cancel{\frac{f''(\eta)}{2}(x^*-x)^2}
\end{equation*}\vspace{0.2cm}

\textsc{Eindimensionales Newton-Verfahren}\\
\begin{equation*}
\Phi : U \rightarrow \mathbb{R} \qquad x \rightarrow x- \frac{f(x)}{f'(x)}
\end{equation*}

\emph{Indem der dritte Term der Taylorreihe $\cancel{\text{wegf\a llt}}$, approximieren wir die Funktion $f$ durch ihre Tangente im Punkt $x$. Die Nullstelle der Tangente ist die n\a chste Iterierte $\Phi(x)$.}\vspace{0.2cm}

\textsc{Konvergenz}\\
Sei $r\in \mathbb{R}_{>0}$ und $U=(x^*-r, x^*+r)$ und gelte $f\in C^2(U)$, $|1/f'(x)|\leq C_1 \forall x\in U$, $|f''(x)| \leq C_2 \forall x\in U$ und $r \leq \frac{2}{C_1C_2}$, dann ist die Abbildung $\Phi$ f\u r das Newton Verfahren eine Selbstabbildung auf $U$, sodass gilt:
\begin{align*}
|\Phi(x) - x^* | \leq \frac{C_1C_2}{2}|x-x^*|^2 \qquad \forall x\in U
\end{align*}

Newton-Verfahren konvergiert, falls $x^{(0)}$ in $U$ liegt. Konvergenz ist umso schneller, je n\a her die Iterierten an der L\o sung liegen. \textbf{Quadratische Konvergenz}.

\subsection{ND-Newton-Verfahren}
\textsc{Hauptsatz der Integral- und Differentialrechnung}\\
\begin{align*}
f(b)-f(a) = \int_a^b f'(t) dt
\end{align*}\vspace{0.2cm}

\textsc{Newton-Verfahren}\\
Sei $U \subseteq \R{n}$ und $Df(\v{x})$ f\u r alle $\v{x} \in U$ regul\a r (also invertierbar). Es gilt:
\begin{align*}
\Phi : U \rightarrow \R{d}, \qquad \v{x} \mapsto \v{x} - Df(\v{x})^{-1}f(\v{x})
\end{align*}\vspace{0.2cm}

\textsc{Konvergenz}\\
Sei $r\in \mathbb{R}_{>0}$ und $U= K(\v{x}^*,r)$ und gelte \\
$f\in C^1(U,\R{b})$,\\
 $\Vert Df(\v{x})^{-1}\Vert_2 \leq C_1 \forall \v{x} \in U$,\\
$\Vert Df(\v{x}) - Df(\v{y})\Vert_2 \leq C_2 \vert \v{x} - \v{y} \Vert_2 \forall \v{x}, \v{y} \in U$ und \\
$r\leq \frac{2}{C_1C_2}$.\\
Dann ist $\Phi$ eine Selbstabbildung auf der Kugel $U$ und es gilt
\begin{align*}
\Vert \Phi(\v{x}) - \v{x}^*\Vert_2 \leq \frac{C_1C_2}{2}\Vert \v{x} - \v{x}^* \Vert_2^2 \qquad \forall \v{x}\in U
\end{align*}
Quadratische Konvergenz unter schw\a cheren Voraussetzungen, denn $f'$ muss Lipschitz-stetig sein.\vspace{0.2cm}

\textsc{Umsetzung}\\
Anstatt Inverse Jacobimatrix zu berechnen, lineares Gleichungssystem nach $\v{d}$ l\o sen (erh\o hte numerische Stabilit\a t):
\begin{align*}
\v{x}^{(m+1)} = \v{x}^{(m)} + \v{d}^{(m)} &\qquad \v{d}^{(m)} = -Df(\v{x}^{(m)})^{-1}f(\v{x}^{(m)})\\
Df(\v{x}^{(m)})\v{d}^{(m)} &= -f(\v{x}^{(m)})
\end{align*}\vspace{0.2cm}

\textsc{Ged\a mpftes Newton-Verfahren}\\
Um Divergenz zu vermeiden Newton-Richtung mit D\a mpfungsparameter $\sigma^{(m)}$ multiplizieren. Sorgt daf\u r, dass der Fehler nicht gr\o \s er als im vorangehenenden Schritt werden kann.
\begin{align*}
\v{x}^{(m+1)} = \v{x}^{(m)} + \sigma^{(m)}\v{d}^{(m)}
\end{align*}

\section{Eigenwertprobleme}
\subsection{Vektoriteration}
\subsection{Inverse Iteration}
\subsection{Orthogonale Iteration}
\subsection{QR-Iteration}
\subsection{Praktische QR-Iteration}

\section{Approximation von Funktionen}
\subsection{Polynominterpolation}
\subsection{Neville-Aitken-Verfahren}
\subsection{Newtons dividierte Differenzen}
\subsection{Approximation von Funktionen}

\section{Numerische Integration}
\subsection{Quadraturformeln}
\subsection{Fehleranalyse}
\clearpage