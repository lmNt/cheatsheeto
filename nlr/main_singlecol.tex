\documentclass[11pt]{article}
\renewcommand{\baselinestretch}{1.05}
\usepackage{amsmath,amsthm,verbatim,amssymb,amsfonts,amscd, graphicx}
\usepackage{graphics, multirow}
\usepackage{xcolor}
\usepackage{color}
\usepackage[makeroom]{cancel} % zum durchstreichen von gleichungen
\topmargin0.0cm
\headheight0.0cm
\headsep0.0cm
\oddsidemargin0.0cm
\textheight23.0cm
\textwidth16.5cm
\footskip1.0cm
\theoremstyle{plain}
\newtheorem{theorem}{Theorem}
\newtheorem{corollary}{Corollary}
\newtheorem{lemma}{Lemma}
\newtheorem{proposition}{Proposition}
\newtheorem*{surfacecor}{Corollary 1}
\newtheorem{conjecture}{Conjecture}
\newtheorem{question}{Question}
\theoremstyle{definition}
\newtheorem{definition}{Definition}
\renewcommand{\a}{\"{a}}
\renewcommand{\o}{\"{o}}
\renewcommand{\u}{\"{u}}
\newcommand{\A}{\"{A}}
\renewcommand{\O}{\"{O}}
\newcommand{\U}{\"{U}}
\newcommand{\s}{\ss}
\let\mbb\boldsymbol
\renewcommand\boldsymbol{\mbb}
% vector, bold face non italic
\newcommand{\R}[1]{\mathbb{R}^{#1}}
\newcommand{\Rr}[2]{\mathbb{R}^{#1 \times #2}}
%\renewcommand{\v}[1]{\mbox{\bf #1}}
% vector, bold face italic
\renewcommand{\v}[1]{\mbb{#1}} % vector, ie bold face

\begin{document}
\begin{center}
     \Large{\textbf{NLR Cheat Cheetos (yumyum)}} \\
\end{center}

\section{LTV Systeme}
\begin{align*}
\vdiff{x}=A(t)\v{x}+B(t)\v{u}\\
\v{y}=C(t)\v{x}+D(t)\v{u}
\end{align*}

\subsection{Transitionsmatrix und L\o sung der Zustandsdifferenzialgleichung}
L\o sung f\u r $\vdiff{x}=A(t)\v{x}$ in der Form $\v{x}(t)=\Phi(t,t_0)\v{x}_0$ mit Transitionsmatrix $\Phi(t,t_0)$ aus der Peano-Baker-Reihe. \\
\textsc{Eigenschaften der Transitionsmatrix:}\\
Anfangswert: $\Phi(t_0,t_0)=E$, Produkteigenschaft: $\Phi(t_2,t_0)=\Phi(t_2,t_1)\Phi(t_1,t_0)$, Invertierbarkeit: $\Phi^{-1}(t,t_0)=\Phi(t_0,t)$, Differenzierbarkeit: $\frac{d}{dt}\Phi(t,t_0)=A(t)\Phi(t,t_0)$, Determinante: $\det\Phi(t,t_0)=\exp\left(\int_{t_0}^{t}\tr A(\tau)\text{d}\tau\right)$.\\
Eigenwerte von $A(t)$ geben keine Aussage \u ber Stabilit\a t.


\subsubsection{Spezialf\a lle geschlossener L\o sungen}
$\dot{x}=a(t)x$ mit L\o sung $x=\exp\left(\int_{t_0}^t a(\tau)\text{d}\tau \right) x_0$. Bei $\dim\v{x}>1$ und $A(t)A(\tau)=A(\tau)A(t)$ dann gilt $\Phi(t,\tau)=\exp\left(\int_{\tau}^t A(s)\text{d}s \right)$

\subsubsection{Periodische Matrizen}
F\u r $\vdiff{x}=A(t)\v{x}$ mit $A(t)=A(t+\omega)$ dann gilt $\Phi(t,\tau)=P(t,\tau)e^{R(t-\tau)}$ mit $P(t,\tau)=P(t+\omega,\tau)$ und $R=\const$

\subsubsection{L\o sung der nicht-autonomen Zustandsdifferenzialgleichung}
L\o sung von
\begin{align*}
\vdiff{x}=A(t)\v{x}+B(t)\v{u}\\
\v{y}=C(t)\v{x}+D(t)\v{u}
\end{align*}
lautet
\begin{align*}
\v{x}&=\Phi(t,t_0)\v{x}_0 + \int_{t_0}^t \Phi(t,\tau)B(\tau)\v{u}(\tau)\text{d}\tau \\
\v{y}&=C(t)\Phi(t,t_0)\v{x}_0+\int_{t_0}^t C(t)\Phi(t,\tau)B(\tau)\v{u}(\tau)\text{d}\tau + D(t)\v{u}
\end{align*}

\subsection{Zustandstransformationen und \a quivalente Systemdarstellungen}
Zustandstransformation $\v{z}=V(t)\v{x}$ mit Transformationsmatrix $V(t)$ regul\a r im betrachteten Zeitintervall und $\dot{V}(t)$ existiert und ist stetig im Intervall.
\begin{align*}
\vdiff{z}&=[\dot{V}(t)+V(t)A(t)]V^{-1}(t)\v{z}+V(t)B(t)\v{u} \\
\v{y}&=C(t)V^{-1}(t)\v{z}(t)+D(t)\v{u}
\end{align*}
$\v{z}=V(t)\v{x}$ ist eine Lyapunov-Transformation, falls $V(t)$ und $V^{-1}(t)$ $\forall t$ beschr\a nkt sind. Dann folgt aus der exponentiellen Stabilit\a t des einen Systems, die exponentielle Stabilit\a t des jeweils anderen Systems.

\subsection{Stabilit\a t}
$\vdiff{x}=A(t)\v{x}$ ist stabil wenn $\Vert \Phi(t,t_0)\Vert \leq M$ mit $M=\const \geq 0$. Das System ist asymptotisch stabil falls $\lim_{t\rightarrow \infty} \Phi(t,t_0)\v{x}_0 \rightarrow 0 \;\; \forall x_0$. Exponentiell stabil falls $\Vert \Phi(t,t_0)\Vert \leq Me^{-\omega(t-t_0)}$

\subsection{Steuerbarkeit und Beobachtbarkeit}
Steuerbar falls $\text{rang}S(A(t),B(t))=n$, mit
\begin{align*}
S(A(t),B(t))=
\begin{bmatrix}
N_A^0B(t) & N_A^1B(t) & \cdots & N_A^{n-1}B(t)
\end{bmatrix}
\end{align*} 
wobei $N_A^kB(t)=N_A(N_A^{k-1}B(t))$ $N_A^1B(t)=-\dot{B}(t)+A(t)B(t), \quad N_A^0B(t)=B(t)$, also $A(t)$ und $B(t)$ m\u ssen entsprechend $(n-2)$ bzw. $(n-1)$-fach stetig diff.bar sein.\\
Beobachtbar falls $\text{rang}O(C(t),A(t))=n$ mit
\begin{align*}
O(C(t),A(t))=\begin{bmatrix}
M_A^0 C(t) & M_A^1C(t) & \cdots & M_A^{n-1}C(t)
\end{bmatrix}^T
\end{align*}
wobei $M_A^kC(t)=M_A(M_A^{k-1}C(t))$ und $M_A^1C(t)=\dot{C}(t)+C(t)A(t)$.

\subsection{Entwurf von Zustandsreglern}
\subsubsection{Eingr\o \ss ensysteme}
In SISO-RNF bringen \u ber $V(t)$ mit Ansatz $z_1=\v{w}^T(t)\v{x}$ und sukzessive Differentiation nach $t$. Eingangsgr\o \ss e $\v{u}$ trifft erst in $z_n(t)$ auf.
\begin{align*}
\v{z}=\begin{bmatrix}
M_A^0\v{w}^T(t) \\
\vdots \\
M_A^{n-1}\v{w}^T(t)
\end{bmatrix}
\v{x}=V(t)\v{x}\\
V(t)\v{b}(t)=\begin{bmatrix}
0 \\
\vdots\\
\tilde{b}_{n-1}(t)
\end{bmatrix}
\end{align*}
Lemma 2.2 besagt f\u r $k\geq 0$: $(M_A^0\v{w}^T(t))\v{v}(t)=0, \cdots, (M_A^k\v{w}^T(t))\v{v}(t)=0$ bzw. $\v{w}^T(t)(N_A^0\v{v}(t)) = 0, \cdots, \v{w}^T(t)(N_A^k\v{v}(t))=0$ und somit gilt:
\begin{align*}
\v{w}^T(t)=\begin{bmatrix}
0 \cdots \tilde{b}_{n-1}(t)
\end{bmatrix}
S^{-1}(A(t),\v{b}(t))
\end{align*}
mit Freiheitsgrad $\tilde{b}_{n-1}(t)$. \\

\textsc{Eigenwertvorgabe mit Ackermann-Formel}\\
Mit $\tilde{\v{a}}=-(M_A^n\v{w}^T(t))V^{-1}(t)$ w\a hle $u=\frac{1}{\tilde{b}_{n-1}(t)}(\tilde{\v{a}}(t)\v{z}+v)$ mit neuem Eingang $v(t)$ als Integratorkette $\dot{z}_1=z_2, \; \cdots, \; \dot{z}_n=v$ (Brunovsky-Normalform). Mit $p^*(\lambda)=\prod_{j=1}^n(\lambda-\lambda_j^*)$ als gew\u nschtes CharPoly w\a hle: $v=-p_0z_1-\cdots-p_{n-1}z_n$ ergibt $u=-\tilde{v{k}}^T(t)\v{z}$. Es folgt f\u r 
\begin{align*}
\vdiff{z}=(\tilde{A}(t)-\tilde{\v{b}}(t)\tilde{\v{k}}^T(t))\v{z}=\begin{bmatrix}
0 & 1 & 0 &\cdots & 0\\
0 & 0 & 1 &\cdots & 0\\
\vdots &  & &  & \vdots \\
0 & 0 & 0 &\cdots & 1\\
-p_0 & -p_1 & -p_2 &\cdots & -p_{n-1}\\
\end{bmatrix} \v{z}
\end{align*}

\subsubsection{Mehrgr\o \ss ensysteme}
Steuerbarkeitsindizes $\rho_j$ einf\u hren. Es gilt
\begin{align*}
S(A(t),B(t))=\\
\begin{bmatrix}
\v{b}_1(t) \cdots \v{b}_m(t) \; \vline \; N_A^1\v{b}_1(t) \cdots N_A^1\v{b}_m(t) \; \vline \; N_A^{n-1}\v{b}_1(t) \cdots  N_A^{n-1}\v{b}_m(t)
\end{bmatrix}
\end{align*}
Reduzierte Steuerbarkeitsmatrix:
\begin{align*}
\bar{S}(A(t),B(t))=
\begin{bmatrix}
\v{b}_1(t) \cdots N_A^{\rho_1-1}\v{b}_1(t)  \; \vline \; \cdots \; \vline \; \v{b}_m(t) \cdots N_A^{\rho_m-1}\v{b}_m(t)
\end{bmatrix}
\end{align*}
\subsection{Entwurf von Zustandsbeobachtern}
\subsubsection{Eingr\o \ss ensysteme}
\subsubsection{Mehrgr\o \ss ensysteme}


\section{Analyse nichtlinearer Systeme}
\subsection{Fluss einer DGL}
\subsection{Existenz und Eindeutigkeit der L\o sung NLDGLSysteme}
\subsection{Zustandsraum nichtlinearer Systeme}
\subsubsection{Mannigfaltigkeiten}
\subsubsection{Tangentialraum und Vektorfeld}
\subsubsection{Zusammenhang zwischen Vektorfeld und Differenzialgleichung}
\subsection{Zustandstransformationen und Diffeomorphismen}
\subsection{Lie-Ableitung und Lie-Klammern}
\subsubsection{Lie-Ableitung}
\subsubsection{Lie-Klammer}
\subsection{Distributionen und Involutivit\a t}
\subsection{Steuerbarkeit und Erreichbarkeit nichtlinearer Systeme}
\subsubsection{Nichtlineare Systeme ohne Driftterm}
\subsubsection{Nichtlineare Systeme mit Driftterm}

\section{Exakte Linearisierung und Flachheit}
\subsection{Exakte Eingangs-/Ausgangslinearisierung f\u r Eingr\o \ss ensysteme}
\subsubsection{Relativer Grad}
\subsubsection{Byrnes-Isidori-Normalform}
\subsubsection{Nulldynamik}
\subsection{Exakte Eingangs-/Zustandslinearisierung f\u r Eingr\o \ss ensysteme}
\end{document}