\begin{center}
     \Large{\textbf{NLR Cheat Cheetos (yumyum)}} \\
\end{center}

\section{LTV Systeme}
\begin{align*}
\vdiff{x}=A(t)\v{x}+B(t)\v{u}\\
\v{y}=C(t)\v{x}+D(t)\v{u}
\end{align*}

\subsection{Transitionsmatrix und L\o sung der Zustandsdifferenzialgleichung}
L\o sung f\u r $\vdiff{x}=A(t)\v{x}$ in der Form $\v{x}(t)=\Phi(t,t_0)\v{x}_0$ mit Transitionsmatrix $\Phi(t,t_0)$ aus der Peano-Baker-Reihe. \\
\textsc{Eigenschaften der Transitionsmatrix:}\\
Anfangswert: $\Phi(t_0,t_0)=E$, Produkteigenschaft: $\Phi(t_2,t_0)=\Phi(t_2,t_1)\Phi(t_1,t_0)$, Invertierbarkeit: $\Phi^{-1}(t,t_0)=\Phi(t_0,t)$, Differenzierbarkeit: $\frac{d}{dt}\Phi(t,t_0)=A(t)\Phi(t,t_0)$, Determinante: $\det\Phi(t,t_0)=\exp\left(\int_{t_0}^{t}\tr A(\tau)\text{d}\tau\right)$.\\
Eigenwerte von $A(t)$ geben keine Aussage \u ber Stabilit\a t.


\subsubsection{Spezialf\a lle geschlossener L\o sungen}
$\dot{x}=a(t)x$ mit L\o sung $x=\exp\left(\int_{t_0}^t a(\tau)\text{d}\tau \right) x_0$. Bei $\dim\v{x}>1$ und $A(t)A(\tau)=A(\tau)A(t)$ dann gilt $\Phi(t,\tau)=\exp\left(\int_{\tau}^t A(s)\text{d}s \right)$

\subsubsection{Periodische Matrizen}
F\u r $\vdiff{x}=A(t)\v{x}$ mit $A(t)=A(t+\omega)$ dann gilt $\Phi(t,\tau)=P(t,\tau)e^{R(t-\tau)}$ mit $P(t,\tau)=P(t+\omega,\tau)$ und $R=\const$

\subsubsection{L\o sung der nicht-autonomen Zustandsdifferenzialgleichung}
L\o sung von
\begin{align*}
\vdiff{x}=A(t)\v{x}+B(t)\v{u}\\
\v{y}=C(t)\v{x}+D(t)\v{u}
\end{align*}
lautet
\begin{align*}
\v{x}&=\Phi(t,t_0)\v{x}_0 + \int_{t_0}^t \Phi(t,\tau)B(\tau)\v{u}(\tau)\text{d}\tau \\
\v{y}&=C(t)\Phi(t,t_0)\v{x}_0+\int_{t_0}^t C(t)\Phi(t,\tau)B(\tau)\v{u}(\tau)\text{d}\tau + D(t)\v{u}
\end{align*}

\subsection{Zustandstransformationen und \a quivalente Systemdarstellungen}
Zustandstransformation $\v{z}=V(t)\v{x}$ mit Transformationsmatrix $V(t)$ regul\a r im betrachteten Zeitintervall und $\dot{V}(t)$ existiert und ist stetig im Intervall.
\begin{align*}
\vdiff{z}&=[\dot{V}(t)+V(t)A(t)]V^{-1}(t)\v{z}+V(t)B(t)\v{u} \\
\v{y}&=C(t)V^{-1}(t)\v{z}(t)+D(t)\v{u}
\end{align*}
$\v{z}=V(t)\v{x}$ ist eine Lyapunov-Transformation, falls $V(t)$ und $V^{-1}(t)$ $\forall t$ beschr\a nkt sind. Dann folgt aus der exponentiellen Stabilit\a t des einen Systems, die exponentielle Stabilit\a t des jeweils anderen Systems.

\subsection{Stabilit\a t}
$\vdiff{x}=A(t)\v{x}$ ist stabil wenn $\Vert \Phi(t,t_0)\Vert \leq M$ mit $M=\const \geq 0$. Das System ist asymptotisch stabil falls $\lim_{t\rightarrow \infty} \Phi(t,t_0)\v{x}_0 \rightarrow 0 \;\; \forall x_0$. Exponentiell stabil falls $\Vert \Phi(t,t_0)\Vert \leq Me^{-\omega(t-t_0)}$

\subsection{Steuerbarkeit und Beobachtbarkeit}
Steuerbar falls $\text{rang}S(A(t),B(t))=n$, mit
\begin{align*}
S(A(t),B(t))=
\begin{bmatrix}
N_A^0B(t) & N_A^1B(t) & \cdots & N_A^{n-1}B(t)
\end{bmatrix}
\end{align*} 
wobei $N_A^kB(t)=N_A(N_A^{k-1}B(t))$ $N_A^1B(t)=-\dot{B}(t)+A(t)B(t), \quad N_A^0B(t)=B(t)$, also $A(t)$ und $B(t)$ m\u ssen entsprechend $(n-2)$ bzw. $(n-1)$-fach stetig diff.bar sein.\\
Beobachtbar falls $\text{rang}O(C(t),A(t))=n$ mit
\begin{align*}
O(C(t),A(t))=\begin{bmatrix}
M_A^0 C(t) & M_A^1C(t) & \cdots & M_A^{n-1}C(t)
\end{bmatrix}^T
\end{align*}
wobei $M_A^kC(t)=M_A(M_A^{k-1}C(t))$ und $M_A^1C(t)=\dot{C}(t)+C(t)A(t)$.

\subsection{Entwurf von Zustandsreglern}
\subsubsection{Eingr\o \ss ensysteme}
In SISO-RNF bringen \u ber $V(t)$ mit Ansatz $z_1=\v{w}^T(t)\v{x}$ und sukzessive Differentiation nach $t$. Eingangsgr\o \ss e $\v{u}$ trifft erst in $z_n(t)$ auf.
\begin{align*}
\v{z}=\begin{bmatrix}
M_A^0\v{w}^T(t) \\
\vdots \\
M_A^{n-1}\v{w}^T(t)
\end{bmatrix}
\v{x}=V(t)\v{x}\\
V(t)\v{b}(t)=\begin{bmatrix}
0 \\
\vdots\\
\tilde{b}_{n-1}(t)
\end{bmatrix}
\end{align*}
Lemma 2.2 besagt f\u r $k\geq 0$: $(M_A^0\v{w}^T(t))\v{v}(t)=0, \cdots, (M_A^k\v{w}^T(t))\v{v}(t)=0$ bzw. $\v{w}^T(t)(N_A^0\v{v}(t)) = 0, \cdots, \v{w}^T(t)(N_A^k\v{v}(t))=0$ und somit gilt:
\begin{align*}
\v{w}^T(t)=\begin{bmatrix}
0 \cdots \tilde{b}_{n-1}(t)
\end{bmatrix}
S^{-1}(A(t),\v{b}(t))
\end{align*}
mit Freiheitsgrad $\tilde{b}_{n-1}(t)$. \\

\textsc{Eigenwertvorgabe mit Ackermann-Formel}\\
Mit $\tilde{\v{a}}=-(M_A^n\v{w}^T(t))V^{-1}(t)$ w\a hle $u=\frac{1}{\tilde{b}_{n-1}(t)}(\tilde{\v{a}}(t)\v{z}+v)$ mit neuem Eingang $v(t)$ als Integratorkette $\dot{z}_1=z_2, \; \cdots, \; \dot{z}_n=v$ (Brunovsky-Normalform). Mit $p^*(\lambda)=\prod_{j=1}^n(\lambda-\lambda_j^*)$ als gew\u nschtes CharPoly w\a hle: $v=-p_0z_1-\cdots-p_{n-1}z_n$ ergibt $u=-\tilde{\v{k}}^T(t)\v{z}$. Es folgt f\u r 
\begin{align*}
\vdiff{z}=(\tilde{A}(t)-\tilde{\v{b}}(t)\tilde{\v{k}}^T(t))\v{z}=\begin{bmatrix}
0 & 1 & 0 &\cdots & 0\\
0 & 0 & 1 &\cdots & 0\\
\vdots &  & &  & \vdots \\
0 & 0 & 0 &\cdots & 1\\
-p_0 & -p_1 & -p_2 &\cdots & -p_{n-1}\\
\end{bmatrix} \v{z}
\end{align*}
Im Originalzustand $\vdiff{x}(t)$ folgt die exp. Stabilit\a t, falls $V(t)$ eine Lyapunov Transformation ist:
\begin{align*}
\vdiff{x}=(A(t)-\v{b}(t)\v{k}^T(t))\v{x}
\end{align*}
mit $\tilde{\v{k}}^T(t)=\v{k}^T(t)V^{-1}(t)$.

\subsubsection{Mehrgr\o \ss ensysteme}
Steuerbarkeitsindizes $\rho_j$ einf\u hren. Es gilt
\begin{align*}
S(A(t),B(t))=\\
\begin{bmatrix}
\v{b}_1(t) \cdots \v{b}_m(t) \; \vline \; N_A^1\v{b}_1(t) \cdots N_A^1\v{b}_m(t) \; \vline \; N_A^{n-1}\v{b}_1(t) \cdots  N_A^{n-1}\v{b}_m(t)
\end{bmatrix}
\end{align*}
Reduzierte Steuerbarkeitsmatrix:
\begin{align*}
\bar{S}(A(t),B(t))=
\begin{bmatrix}
\v{b}_1(t) \cdots N_A^{\rho_1-1}\v{b}_1(t)  \; \vline \; \cdots \; \vline \; \v{b}_m(t) \cdots N_A^{\rho_m-1}\v{b}_m(t)
\end{bmatrix}
\end{align*}
Anleitung:
\begin{itemize}
\item Beginne mit $\v{b}_1(t)$. Berechne $N_A\v{b}_1(t),\cdots,N_A^{\rho_1-1}\v{b}_1(t)$. Abbruch, wenn $N_A^{\rho_1}\v{b}_1(t)$ linear abh\a ngig von den vorherigen Vektoren der Sequenz.
\item Falls $\rho_1\neq n$ nutze $\v{b}_2(t)$ mit der Sequenz $N_A\v{b}_2(t),\cdots,N_A^{\rho_2}\v{b}_2(t)$ bis linear abh\a ngig von den vorherigen Sequenzen.
\item Es muss gelten: $\sum_{j=1}^m \rho_j=n$
\end{itemize}
Die MIMO-RNF kann mit $\v{z}(t)=V(t)\v{x}(t)$ erreicht werden wenn $\bar{S}(A(t),B(t))$ regul\a r ist, mit 
\begin{align*}
V(t)=\begin{bmatrix}
M_A^0\v{w}_1^T(t)\\
\vdots\\
M_A^{\rho_1-1}\v{w}_1^T(t)\\
\hline\\
\cdots\\
\hline\\
M_A^0\v{w}_m^T(t)\\
\vdots\\
M_A^{\rho_m-1}\v{w}_m^T(t)\\
\end{bmatrix}
\end{align*} 
und 
\begin{align*}
&\begin{bmatrix}
\v{w}_1^T(t) \\
\vdots\\
\v{w}_m^T(t)
\end{bmatrix}
=\\
&\begin{bmatrix}
0      \cdots  \tilde{b}_{1,\rho_1-1}(t) & \vline & \cdots & \vline & 0      \cdots  \tilde{b}_{1,\rho_m-1}(t) \\
\vdots                                   & \vline &        & \vline & \vdots          \\
0      \cdots  \tilde{b}_{m,\rho_1-1}(t) & \vline & \cdots & \vline & 0      \cdots  \tilde{b}_{m,\rho_m-1}(t)\\
\end{bmatrix} \bar{S}^{-1}(\cdot)
\end{align*}
mit $\tilde{b}_{j,\rho_j-1}(t)$ als Freiheitsgrade. Kompakte Darstellung \u ber $\zeta=\v{w}_j^T(t)\v{x}$ und Ausnutzung der kanonischen Form ergibt
\begin{align*}
&\begin{bmatrix}
\frac{d^{\rho_1}}{dt^{\rho_1}}\zeta_1\\
\vdots\\
\frac{d^{\rho_m}}{dt^{\rho_m}}\zeta_m
\end{bmatrix}
=
\underbrace{\begin{bmatrix}
M_A^{\rho_1}\v{w}_1^T(t) \\
\vdots \\
M_A^{\rho_m}\v{w}_m^T(t) 
\end{bmatrix}}_{\mathcal{A}(t) \in \Rr{m}{n}}
\v{x}\\ &+
\underbrace{
\begin{bmatrix}
(M_A^{\rho_1-1}\v{w}_1^T(t))\v{b}_1(t) \cdots  (M_A^{\rho_1-1}\v{w}_1^T(t))\v{b}_m(t) \\
\vdots   \\
(M_A^{\rho_m-1}\v{w}_m^T(t))\v{b}_1(t)  \cdots  (M_A^{\rho_m-1}\v{w}_m^T(t))\v{b}_m(t) 
\end{bmatrix}}_{\text{Kopplungsmatrix } \mathcal{K}(t) \in \Rr{m}{m}}
\v{u}
\end{align*}
\begin{align*}
K(t)&=\mathcal{K}^{-1}(t)
\begin{bmatrix}
(p_{1,0}M_A^0+\cdots+p_{1,\rho_1-1}M_A^{\rho_1-1}+M_A^{\rho_1})\circ \v{w}_1^T(t) \\
\vdots\\
(p_{m,0}M_A^0+\cdots+p_{m,\rho_m-1}M_A^{\rho_m-1}+M_A^{\rho_m})\circ \v{w}_m^T(t) \\
\end{bmatrix}
\\&=
\begin{bmatrix}
\tilde{\v{k}}_1^T(t)\\
\vdots\\
\tilde{\v{k}}_m^T(t)\\
\end{bmatrix}
V(t)
\end{align*}
Zustandsr\u ckf\u hrung:
\begin{align*}
\v{u}=-K(t)\v{x} = \mathcal{K}^{-1}(t)(-\mathcal{A}(t)\v{x}+\v{v})
\end{align*}
und $m$ entkoppelte Integratorketten der L\a nge $\rho_j$ in den Koordinaten $\zeta_j(t)$ erhalten.\\
Originalkoordinaten
\begin{align*}
\vdiff{x}=(A(t)-B(t)K(t))\v{x}
\end{align*}

\subsection{Entwurf von Zustandsbeobachtern}
Entwurf eines (vollst\a ndigen) Luenberger-Beobachters:
\begin{align*}
\vhatdiff{x} = \underbrace{A(t)\hat{\v{x}}+B(t)\v{u}}_{\text{Simulator}}+\underbrace{L(t)(\v{y}-\hat{\v{y}})}_{\text{Korrektur}}\\
\hat{\v{y}}=C(t)\hat{\v{x}}+D(t)\v{u}
\end{align*}
mit Beobachterfehlerdynamik $\tilde{\v{x}}(t)=\v{x}(t)-\hat{v{x}}(t)$ ergibt sich 
\begin{align*}
\vtildediff{x}=\underbrace{(A(t)-L(t)C(t))}_{=A_b(t)}\tilde{\v{x}}
\end{align*}
\subsubsection{Eingr\o \ss ensysteme}
\textsc{Ackermann-Formel}\\
In SISO-BNF \u berf\u hren mit Transformation $\v{z}=V(t)\v{x}(t)$, wenn $O(\cdot)$ regul\a r:
\begin{align*}
V^{-1}(t)=
\begin{bmatrix}
N_A^0\v{v}(t) & \cdots & N_A^{n-1}\v{v}(t)
\end{bmatrix}\\
\v{v}(t)=O^{-1}(\v{c}^T(t),A(t))\begin{bmatrix}
0\\
\vdots\\
\tilde{c}_{n-1}(t)
\end{bmatrix}
\end{align*}
mit Freiheitsgrad $\tilde{c}_{n-1}(t)$.
Zeitvariante Beobachterkorrektur
\begin{align*}
\v{l}(t)=\frac{1}{\tilde{c}_{n-1}(t)} \left[ p_0N_A^0 + \cdots + p_{n-1}N_A^{n-1}+N_A^n \right] \circ \v{v}(t)
\end{align*}
f\u hrt auf zeitinvariante Dynamikmatrix $\tilde{A}_b = \tilde{A}(t) -\tilde{\v{l}}(t)\tilde{\v{c}}^T(t)$ mit $\tilde{\v{l}}(t)=V(t)\v{l}(t)$
Wenn die Transformation $\v{z}(t)$ eine Lyapunov-Transformation ist, dann folgt die exp. Stabilit\a t von
\begin{align*}
\vtildediff{x}=(A(t)-\v{l}(t)\v{c}^T(t))\tilde{\v{x}}
\end{align*}


\subsubsection{Mehrgr\o \ss ensysteme}
Reduzierte Beobachtbarkeitsmatrix zusammengesetzt aus den $n$ lin.unabh. Zeilenvektoren von $O(\cdot)$:
\begin{align*}
\bar{O}(C(t),A(t))=\begin{bmatrix}
\v{c}_1^T(t) \\
\vdots \\
M_A^{\rho_1-1}\v{c}_1^T(t) \\
\hline \\
\vdots\\
\hline\\
\v{c}_p^T(t) \\
\vdots\\
M_A^{\rho_p-1}\v{c}_p^T(t)
\end{bmatrix},
\quad \sum_{j=1}^p\rho_j=n
\end{align*}

\textsc{Trafo in MIMO-BNF}\\
Durch 
\begin{align*}
V^{-1}(t)=\begin{bmatrix}
N_A^0\v{v}_1(t) \cdots N_A^{\rho-1}\v{v}_1(t) \vline \cdots \vline N_A^0\v{v}_p(t) \cdots N_A^{\rho_p-1}\v{v}_p(t)
\end{bmatrix}
\end{align*}
und
\begin{align*}
\begin{bmatrix}
\v{v}_1(t) \cdots \v{v}_p(t)
\end{bmatrix}
=\bar{O}^{-1}(\cdot)\bar{C}^T(t)
\end{align*}
mit 
\begin{align*}
\bar{C}^T(t)=
\begin{bmatrix}
0 & \cdots & 0 \\
\vdots & & \vdots \\
0 & \cdots & 0 \\
\tilde{c}_{1,\rho_1-1}(t) & \cdots & \tilde{c}_{p,\rho_1-1}(t) \\
\hline  \\
\cdots & & \cdots \\
\hline \\
0 & \cdots & 0 \\
\vdots & & \vdots \\
0 & \cdots & 0 \\
\tilde{c}_{1,\rho_p-1}(t) & \cdots & \tilde{c}_{p,\rho_p-1}(t) \\
\end{bmatrix}
\end{align*}


\section{Analyse nichtlinearer Systeme}
\begin{align*}
\vdiff{x}=\v{f}(\v{x},t) \quad \text{bzw.} \quad \vdiff{x}=\v{f}(\v{x})
\end{align*}
\subsection{Fluss einer DGL}
\begin{align*}
\v{x}=\Phi_t(\v{x}_0)
\end{align*}
mit $\v{x}(0)=\Phi_0(\v{x})$ folgt dass $\Phi_0(\cdot)=E$. Axiome: $\v{x}(t+\tau)=\Phi_\tau(\v{x}(t))$, $\v{x}(t+\tau)=\Phi_{t+\tau}(\v{x}_0)=\Phi_\tau(\Phi_t(\v{x}_0))$. Transitionseigenschaft:\\
\begin{align*}
\Phi_\tau \circ \Phi_t =\Phi_{t+\tau} \quad\quad \text{(Komposition)}
\end{align*}

\subsection{Existenz und Eindeutigkeit der L\o sung (NL) DGLSysteme}
\textsc{Lokale Existenz und Eindeutigkeit}\\
Sei $\v{f}(\v{x},t)$ st\u ckweise stetig und erf\u lle die Lipschitz Bedingung $\forall t \in [t_0,t_0+\tau]$. Dann existiert ein $\delta>0$, so dass das System genau eine L\o sung f\u r $t\in [t_0,t_0+\delta]$ besitzt (Lokal Lipschitz). Gilt die Lipschitz-Bedingung f\u r alle $\v{x}_1,\v{x}_2$ dann ist $\v{f}(\cdot)$ global Lipschitz.\vspace{0.1cm}

\textsc{Stetigkeit und Lipschitz-Bed.}\\
Sind $\v{f}(\cdot)$ und $\frac{\partial}{\partial\v{x}}\v{f}(\cdot)$ auf der Menge $\mathcal{B} \times [t_0,t_0+\delta]$ stetig, dann ist die L-B erf\u llt.

\textsc{Globale Existenz und Eindeutigkeit}\\
Sei $\v{f}$ st\u ckweise stetig in $t$ und global Lipschitz in $[t_0,t_0+\delta]$, dann besitzt die DGL eine eindeutige L\o sung auf dem Intervall. Sind $\v{f}$ und $\frac{\partial}{\partial\v{x}}\v{f}$ in dem Intervall stetig, dann ist $\v{f}$ genau dann global Lipschitz, wenn $\frac{\partial}{\partial\v{x}}\v{f}$ im Intervall gleichm\a \ss ig beschr\a nkt (induz. Matrixnorm) ist.

\textsc{Satz 3.4}\\


\subsection{Zustandsraum nichtlinearer Systeme}
NKR
\subsubsection{Mannigfaltigkeiten}
NKR
\subsubsection{Tangentialraum und Vektorfeld}
NKR
\subsubsection{Zusammenhang zwischen Vektorfeld und Differenzialgleichung}
NKR

\subsection{Zustandstransformationen und Diffeomorphismen}
TODO
\subsection{Lie-Ableitung und Lie-Klammern}
\subsubsection{Lie-Ableitung}
Lie-Ableitung $L_{\v{f}}h$ bezeichnet die Richtungsableitung einer glatten Funktion $h$ in Richtung eines Vektorfeldes $\v{f}$ am Punkt $\v{p}$:
\begin{align*}
(L_{\v{f}}h)(\v{p})=(\v{f}(\v{p}))(h)=\frac{d}{dt}h(\v{p}+t\v{f})\vert_{t=0} = \frac{d}{dt}h(\v{\sigma}(t))\vert_{t=0}
\end{align*}
In Originalkoordinaten: 
\begin{align*}
(L_{\v{f}}h)(\v{x})=\frac{\partial h}{\partial \v{x}}\v{f}(\v{x})
\end{align*}
Au\ss erdem:
\begin{align*}
L_{\v{g}}L_{\v{f}}h= \frac{\partial L_{\v{f}}h}{\partial\v{x}}\v{g}
\end{align*}
\subsubsection{Lie-Klammer}
\begin{align*}
\text{ad}_{\v{f}}\v{g}= [\v{f},\v{g}]=\frac{\partial \v{g}}{\partial \v{x}}(\v{x})\v{f}(\v{x})-\frac{\partial \v{f}}{\partial \v{x}}(\v{x})\v{g}(\v{x})
\end{align*}
mit Eigenschaften:
\begin{align*}
\begin{array}{cc}
[\v{f},\v{g}]=-[\v{g},\v{f}] & \text{(Schiefsymmetrie)} \\
\text{todo}
\end{array}
\end{align*}
\subsection{Distributionen und Involutivit\a t}
\textsc{Distribution}\\
Eine Vorschrift, die jedem Punkt $\v{p}$ einen linearen Unterraum $\Delta_p$ des Tangentialraumes $\mathcal{T}_p\mathcal{M}$ in der Form $\Delta_p=\text{span}\{\v{v}_{1,\v{p}}, /cdots, \v{v}_{d,\v{p}}\}$ zuordnet wird als (glatte) Distribution bezeichnet. Sie ist regul\a r in einer Umgebung $V$, wenn f\u r alle $\v{q}\in V$ gilt: $\dim(\Delta_{\v{q}})=d$.\vspace{0.2cm}

\textsc{Involutivit\a t}\\
Eine regul\a re Distribution ist dann involutiv auf $V$, wenn f\u r alle $\v{q}\in V$ gilt: $[\v{v}_{j,\v{q}},\v{v}_{k,\v{q}}]\in \Delta_{\v{q}} \quad \forall j,k=1,...,d$. Die Lie-Klammer jeder Kombination von Vektorfeldern muss wieder als Linearkombination dieser Vektorfelder darstellen lassen.

\subsection{Steuerbarkeit und Erreichbarkeit nichtlinearer Systeme}
\textsc{Lokal steuerbar}\\
Lokal steuerbar am Punkt $\v{x}_1\in \mathcal{M}$ wenn eine offene Umgebung $V$ um $\v{x}_1$ existiert, sodass f\u r alle $\v{x}_2\in V$ eine Zeit $\tau$ und ein $\v{u}$ so existieren, dass $\v{x}(\tau)=\Phi_{\tau}(\v{x}_1)$. \\
\textsc{Global steuerbar}\\
Wie lokal, nur f\u r $\v{x}_2 \in \mathcal{M}$.\\
Einerseits: Steuerbarkeit um eine Ruhelage bzw. AP linearisierte System impliziert die lokale Steuerbarkeit des NL Systems.\\
Andererseits: Linearisierung eines NL Systems kann zum Verlust der Steuerbarkeit f\u hren (z.B. kin. Fahrzeugmodell)

\subsubsection{Nichtlineare Systeme ohne Driftterm}
\begin{align*}
\vdiff{x}=\sum_{j=1}^m \v{g}_j(\v{x})u_j
\end{align*}
Genau dann lokal steuerbar um $\v{x}_0$, wenn die Steuerbarkeitsdistribution 
\begin{align*}
\Delta_S(\v{x})=\text{span}\{\v{g}_1, \cdots, \v{g}_m, [\v{g}_i,\v{g}_j], [\v{g}_i,[\v{g}_j,\v{g}_k]],\cdots\}
\end{align*}
mit $i,j,k,...=1,...,m$ die Bedingung $\dim(\Delta_S(\v{x}))=n$ erf\u llt.

\subsubsection{Nichtlineare Systeme mit Driftterm}
\begin{align*}
\vdiff{x}=\v{f}(\v{x})+\sum_{j=1}^m \v{g}_j(\v{x})u_j
\end{align*}

\textsc{Lokale Erreichbarkeit}\\
Das System ist lokal erreichbar am Punkt $\v{x}_1$, wenn eine Umgebung $V$ um $\v{x}_1$, so dass f\u r alle $\v{x}_2 \in V$ eine Zeit $\tau$ und ein $\v{u}$ existieren, dass gilt $\v{x}(\tau)=\Phi_{\tau}(\v{x}_1)=\v{x}_2$.
Genau dann lokal erreichbar um $\v{x}_0$, wenn die Erreichbarkeitsdistribution 
\begin{align*}
\Delta_E(\v{x})=\text{span}\{\v{g}_1, \cdots, \v{g}_m, [\v{g}_i,\v{g}_j], [\v{g}_i,[\v{g}_j,\v{g}_k]],\cdots\}
\end{align*}
mit $i,j,k...=0,...,m$ und $\v{g}_0(\v{x})=\v{f}(\v{x})$ die Bedingung $\dim(\Delta_E(\v{x}))=n$.\vspace{0.2cm}

\textbf{Lokale Erreichbarkeit schw\a chere Eigenschaft als die lokale Steuerbarkeit.}


\section{Exakte Linearisierung und Flachheit}
\begin{align*}
\vdiff{x}=\v{f}(\v{x})+\sum_{j=1}^m \v{g}_j(\v{x})u_j \\
y_j = h_j(\v{x})
\end{align*}
mit glatten Vektorfeldern $\v{f}, \v{g}_i \in \mathcal{TM}$ und glatten Funktionen $h_i(\v{x})$.

\subsection{Exakte Eingangs-/Ausgangslinearisierung f\u r Eingr\o \ss ensysteme}
\begin{align*}
\vdiff{x}=\v{f}(\v{x})+\v{g}(\v{x})u\\
y=h(\v{x})
\end{align*}

\subsubsection{Relativer Grad}
Zeitliche \A nderung des Ausgangs als lineares System mit neuem Eingang $\dot{y}=v$ mit der Zustandsr\u ckf\u hrung:
\begin{align*}
u=\frac{-L_{\v{f}}h(\v{x})+v}{L_{\v{g}}h(\v{x})}
\end{align*}

\textsc{Definition}\\
Das System hat den relativen Grad $r$ an der Stelle $\bar{\v{x}}$ wenn \textbf{(i)} $L_{\v{g}}L_{\v{f}}^kh(\v{x})=0$ f\u r $k=0,...,r-2$ und \textbf{(ii)} $L_{\v{g}}L_{\v{f}}^{r-1}h(\bar{\v{x}})\neq 0$. Existieren Punkte $\v{x}$ sodass $L_{\v{g}}L_{\v{f}}^{r-1}h(\v{x}) = 0$, so ist der RG nicht wohl definiert.\\
\emph{Differenziere den Ausgang so oft, bis der Eingang erstmalig explizit auftritt.}

Relativer Grad eines LTI Eingr\o \ss ensystems entspricht dem Differenzgrad von Z\a hler- und Nennerpolynom der entsprechenden \U bertragungsfunktion (Laurent-Reihe).

\subsubsection{Byrnes-Isidori-Normalform}
Kette von Differentiationen aus RG Def. $\xi_1=y,...,\xi_r=y^{(r-1)}$ f\u hrt auf 
\begin{align*}
y^{(r)}&=L_{\v{f}}^rh(\v{x})+L_{\v{g}}L_{\v{f}}^{r-1}h(\v{x})u \\
\Leftrightarrow u&= \frac{-L_{\v{f}}^rh(\v{x})+v}{L_{\v{g}}L_{\v{f}}^{r-1}h(\v{x})} \quad \text{ mit $y^{(r)}=v$ }
\end{align*}
lineares E/A Verhalten in Form einer Integratorkette der L\a nge $r$.

\textsc{Zustandstransformation auf Byrnes-Isidori-Normalform}\\
\begin{align*}
\v{z}=\begin{bmatrix}
\v{\xi}\\
\v{\eta}
\end{bmatrix}=
\begin{bmatrix}
\xi_1 \\
\vdots \\
\xi_r \\ \hline
\eta_1 \\
\vdots \\
\eta_{n-r}
\end{bmatrix}
=\v{\Phi}(\v{x})=
\begin{bmatrix}
h(\v{x})\\
L_{\v{f}}h(\v{x})\\
\vdots\\
L_{\v{f}}^{r-1}h(\v{x})\\ \hline
\Phi_{r+1}(\v{x})\\
\vdots\\
\Phi_n(\v{x})
\end{bmatrix}
\end{align*}
Es kann $\Phi_{r+1},...,\Phi_n$ so gew\a hlt werden, dass $L_{\v{g}}\Phi_k(\v{x})=0=q_k(\v{\xi},\v{\eta}), \quad k=r+1,...,n$.\\
Diffeomorphismus: $\Phi^{-1}(\v{z})=\v{x}$ (zumindest lokal)%

\begin{align*}
&\Sigma_1: 
\begin{cases}
\dot{\xi}_1=\xi_2 \\
\dot{\xi}_2=\xi_3 \\
\vdots\\
\dot{\xi}_r=a(\v{\xi},\v{\eta})+b(\v{\xi},\v{\eta})u \\
\end{cases}\\
&\Sigma_2: 
\begin{cases}
\dot{\eta}_1=p_1(\v{\xi},\v{\eta})+q_1(\v{\xi},\v{\eta})u \\
\dot{\eta}_2=p_2(\v{\xi},\v{\eta})+q_2(\v{\xi},\v{\eta})u \\
\vdots\\
\dot{\eta}_{n-r}=p_{n-r}(\v{\xi},\v{\eta})+q_{n-r}(\v{\xi},\v{\eta})u \\
\end{cases}\\
&y=\xi_1
\end{align*}
mit
\begin{align*}
a(\v{\xi},\v{\eta})&=L_{\v{f}}^rh\circ \v{\Phi}^{-1}(\v{z})\\
b(\v{\xi},\v{\eta})&=L_{\v{g}}L_{\v{f}}^{r-1}h\circ \v{\Phi}^{-1}(\v{z})\\
p_k(\v{\xi},\v{\eta})&=L_{\v{f}}\Phi_{r+k}\circ \v{\Phi}^{-1}(\v{z})\\
q_k(\v{\xi},\v{\eta})&=L_{\v{g}}\Phi_{r+k}\circ \v{\Phi}^{-1}(\v{z}) \quad\qquad k=1,...,n-r
\end{align*}

\textsc{Exakte E/A-Linearisierung}\\
\begin{align*}
u=\frac{-\v{a}(\v{\xi},\v{\eta})+v}{\v{b}(\v{\xi},\v{\eta})}\\
v=y^{*(r)}-\sum_{j=0}^{r-1}p_j\left(y^{(j)}-y^{*(j)}\right)
\end{align*}
\subsubsection{Nulldynamik}
\emph{Wie muss $\v{x}_0$ und $u(t)$ gew\a hlt werden, damit der Ausgang $y(t)$ $\forall t$ identisch Null ist.}\\
Damit $\dot{\xi}_r=0$, muss $0=a(\v{0},\v{\eta})+b(\v{0},\v{\eta})u$ bzw. $u=-\frac{a(\v{0},\v{\eta})}{b(\v{0},\v{\eta})}$ gelten und es ergibt sich die \textbf{Nulldynamik}:
\begin{align*}
\vdiff{\eta}=\v{p}(\v{0},\v{\eta})
\end{align*}

\textsc{Stabilit\a t der Nulldynamik}\\
Stabilit\a t der Nulldynamik ist entscheidend f\u r die Anwendung der Exakten E/A-Linearisierung.

\textsc{Stbilisierung mit der Exakten E/A-Lin.}\\
Ist die Nulldynamik des Systems lokal asymptotisch (expon.) stabil, dann stabiliert das nl Regelgesetz:
\begin{align*}
u=\frac{1}{L_{\v{g}}L_{\v{f}}^{r-1}h(\v{x})}\left(-L_{\v{f}}^rh(\v{x})+\textcolor{red}{y^{*(r)}}-\sum_{j=0}^{r-1}p_j \left[ L_{\v{f}}^jh(\v{x})\textcolor{red}{-y^{*(j)}}\right]\right)
\end{align*}
mit $p_j$ den Koeffizienten eines Hurwitz-Polynoms das System lokal asymptotisch (expon.).

\subsection{Exakte Eingangs-/Zustandslinearisierung f\u r Eingr\o \ss ensysteme}
Wenn $r=n$ Exakte E/A-Linearisierung $\Rightarrow$ Exakte E/Z-Linearisierung.\\

\textsc{Brunovsky-Normalform}
Bezeichne $y=\lambda(\v{x})$ als fiktive Ausgangsgr\o \ss e mit relativem Grad $r=n$. Dann reduziert sich die Zustandstransformation zu
\begin{align*}
\v{z}=\begin{bmatrix}
z_1 \\
\vdots \\
z_n
\end{bmatrix}=\v{\Phi}(\v{x})=
\begin{bmatrix}
\lambda(\v{x})\\
L_{\v{f}}\lambda(\v{x})\\
\vdots\\
L_{\v{f}}^{n-1}\lambda(\v{x})
\end{bmatrix}
=
\begin{bmatrix}
y\\
\dot{y}\\
\vdots\\
y^{(n-1)}
\end{bmatrix}
\end{align*}
mit \textbf{(i)} $L_{\v{g}}L_{\v{f}}^k \lambda(\v{x})=0$ f\u r $k=0,...,n-2$ und \textbf{(ii)} $L_{\v{g}}L_{\v{f}}^{n-1} \lambda(\v{x})\neq 0$ ergibt vereinfachte Byrnes-Isidori-Normalform:
\begin{align*}
\dot{z}_1&=z_2 \\
\dot{z}_2&=z_3\\
\vdots\\
\dot{z}_n&=a(\v{z})+b(\v{z})u \\
y&=z_1
\end{align*}
mit 
\begin{align*}
a(\v{z})=L_{\v{f}}^n\lambda(\v{x})  (\circ \v{\Phi}^{-1}(\v{z})) &\quad b(\v{z})=L_{\v{g}}L_{\v{f}}^{n-1}\lambda(\v{x})  (\circ \v{\Phi}^{-1}(\v{z}))\\
u&=\frac{-a(\v{z})+v}{b(\v{z})}
\end{align*}

$L_{\v{g}}L_{\v{f}}\lambda(\v{x})$ in ein System von partiellen DGL 1. Ordnung vom Frobenius Typ \u berf\u hren:
\begin{align*}
L_{\v{g}}\lambda(\v{x})=0\\ L_{\text{ad}_{\v{f}}\v{g}}\lambda(\v{x})=0\\ \vdots\\ L_{\text{ad}^{n-2}_{\v{f}}\v{g}}\lambda(\v{x})=0\\ L_{\text{ad}^{n-1}_{\v{f}}\v{g}}\lambda(\v{x})\neq 0 \\
\frac{\partial \lambda(\v{x})}{\partial \v{x}}
[\v{g} \; \text{ad}_{\v{f}}\v{g} \; \cdots \; \text{ad}^{n-2}_{\v{f}}\v{g} \; \text{ad}^{n-1}_{\v{f}}\v{g}] &= [0 \; \cdots \; 0 \; \beta(\v{x})]
\end{align*}

\textsc{Existenz eines Ausgangs mit relativem Grad $r=n$}\\
Es existiert genau eine L\o sung $\lambda(\v{x})$ des Systems von PDGL 1. Ordnung wenn \textbf{(i)} $P_E(\v{x})=[\v{g}(\v{x}),\text{ad}_{\v{f}}\v{g}(\v{x}),...,\text{ad}_{\v{f}}^{n-1}\v{g}(\v{x})]$ Rang $n$ besitzt und \textbf{(ii)} die Distribution $\Delta_E(\v{x})=\text{span}\{\v{g}(\v{x}),\text{ad}_{\v{f}}\v{g}(\v{x}),...,\text{ad}_{\v{f}}^{n-2}\v{g}(\v{x})\}$ involutiv ist in einer Umgebung von $\bar{\v{x}}$.

\subsubsection{Theorem von Frobenius}
Eine regul\a re Distribution ist genau dann vollst\a ndig integrabel, wenn sie involutiv ist.

\subsubsection{Stabilisierung mit Zustandsr\u ckf\u hrung}
Regelgesetz w\a hlen als:
\begin{align*}
u=\frac{1}{b(\v{z})}\left(-a(\v{z})\underbrace{-\sum_{j=0}^{n-1}p_jz_{j+1}}_{v}\right)
\end{align*}
ergibt Matrix mit $p$ und unterster zeile blabla..


\subsection{Erweiterung auf Mehrgr\o \ss ensysteme}
\begin{align*}
\vdiff{x}=\v{f}(\v{x})+\sum_{j=1}^m\v{g}_j(\v{x})u_j\\
y_i=h_i(\v{x}) \quad i=1,...,p
\end{align*}
Annahme: $\dim \v{y}=p=m=\dim \v{u}$

\subsubsection{Exakte E/A-Linearisierung}
\textsc{Vektorieller relativer Grad}\\
Das System hat den vektoriellen relativen Grad $\{r_1,...,r_m\}$ an der Stelle $\bar{x}$, wenn
\textbf{(i)} $L_{\v{g}_j}L_{\v{f}}^kh_i(\v{x})=0, \quad i,j=1,...,m$ f\u r $k=0,1,...,r_i-2$ und 
\textbf{(ii)} die $(m\times m)$-Entkopplungsmatrix $B(x)=\begin{bmatrix}
L_{\v{g}_1}L_{\v{f}}^{r_1-1}h_1(\v{x}) & \cdots & L_{\v{g}_m}L_{\v{f}}^{r_1-1}h_1(\v{x})\\
\vdots & & \vdots \\
L_{\v{g}_1}L_{\v{f}}^{r_m-1}h_m(\v{x}) & \cdots &  L_{\v{g}_m}L_{\v{f}}^{r_m-1}h_m(\v{x})
\end{bmatrix}
$ \\
am Punkt $\v{x}=\bar{\v{x}}$ regul\a r ist.

\textsc{Byrnes-Isidori-NF und Nulldynamik}\\
Sei $r\leq n$ die Summe der einzelnen relativen Grade. Ist $r<n$ dann existieren $(n-r)$ Funktionen $\Phi(\v{x})$ so dass mit
\begin{align*}
\v{z}=\begin{bmatrix}
\xi_{1,1}\\
\vdots\\
\xi_{1,r_1}\\\hline
\vdots\\\hline
\xi_{m,1}\\
\vdots\\
\xi_{m,r_m}\\\hline
\eta_1\\
\vdots\\
\eta_{n-r}
\end{bmatrix}=\v{\Phi}(\v{x})=
\begin{bmatrix}
h_1(\v{x})\\
\vdots\\
L_{\v{f}}^{r_1-1}h_1(\v{x})\\\hline
\vdots\\\hline
h_m(\v{x})\\
\vdots\\
L_{\v{f}}^{r_m-1}h_m(\v{x})\\\hline
\Phi_{r+1}(\v{x})\\
\vdots\\
\Phi_n(\v{x})
\end{bmatrix}
\end{align*}
ein lokaler Diffeomorphismus gegeben ist.
Die Funktionen $\Phi_{r+1}(\v{x}),...,\Phi_n(\v{x})$ k\o nnen \textbf{nicht} wie im Eingr\o \ss enfall zu $L_{\v{g}_j}\Phi_k(\v{x})=0$ gew\a hlt werden, \textbf{au\ss er} wenn $\Delta_0=\text{span}\{\v{g}_1,...\v{g}_m\}$ um $\bar{\v{x}}$ involutiv ist.
BI-NF
\begin{align*}
&\Sigma_1:\begin{cases}
\dot{\xi}_{m,1}=\xi_{m,2}\\
\dot{\xi}_{m,2}=\xi_{m,3}\\
\vdots \\
\dot{\xi}_{m,r_m}=a_m(\v{\xi},\v{\eta})+ \v{b}_m^T(\v{\xi},\v{\eta})\v{u}
\end{cases}\\
&\Sigma_2: \begin{cases}
\dot{\eta}_{1}=p_{1}(\v{\xi},\v{\eta})+ \v{q}_{1}^T(\v{\xi},\v{\eta})\v{u}\\
\dot{\eta}_{2}=p_{2}(\v{\xi},\v{\eta})+ \v{q}_{2}^T(\v{\xi},\v{\eta})\v{u}\\
\vdots \\
\dot{\eta}_{n-r}=p_{n-r}(\v{\xi},\v{\eta})+ \v{q}_{n-r}^T(\v{\xi},\v{\eta})\v{u}
\end{cases}\\
&\v{y}=[\xi_{1,1} \; \xi_{2,1} \; \cdots \; \xi_{m,1}]^T
\end{align*}
mit
\begin{align*}
B(\v{\xi},\v{\eta})=\begin{bmatrix}
\v{b}^T_1(\v{\xi},\v{\eta})\\
\vdots\\
\v{b}_m^T(\v{\xi},\v{\eta})
\end{bmatrix}=B(\v{x})\circ\v{\Phi}^{-1}(\v{z})
\end{align*}
\begin{align*}
\v{u}=B^{-1}(\v{\xi},\v{\eta})(-\v{a}(\v{\xi},\v{\eta})+\v{v})
\end{align*}
\emph{Stabilit\a t des Gesamtsystems bleibt von der Stabilit\a t der entsprechenden Nulldynamik abh\a ngig}.

\textsc{Stabilisierung und Ausgangsregelung}\\
\begin{align*}
\v{u}=B^{-1}(\v{x})
\begin{bmatrix}
-L_{\v{f}}^{r_1}h_1(\v{x})\textcolor{red}{+y_1^{*(r_1)}}-\sum_{l=0}^{r_1-1}p_{1,l} \left[L_{\v{f}}^lh_1(\v{x}) \textcolor{red}{-y_1^{*(l)}} \right]\\
\vdots \\
-L_{\v{f}}^{r_m}h_m(\v{x})\textcolor{red}{+y_m^{*(r_m)}}-\sum_{l=0}^{r_m-1}p_{m,l} \left[L_{\v{f}}^lh_m(\v{x}) \textcolor{red}{-y_m^{*(l)}} \right]
\end{bmatrix}
\end{align*}



\subsubsection{Exakte E/Z-Linearisierung}
BLABLA


\subsection{Differenzielle Flachheit}
\emph{S\a mtliche Zustands und Eingangsgr. werden durch einen flachen Ausgang und dessen Zeitableitungen parametriert.}


\subsubsection{Flache Systeme}
\textsc{Flacher Ausgang}\\
Fiktiver Ausgang $\v{z}(t)$ muss folgende Bedingungen erf\u llen, um ein flacher Ausgang (flaches System) zu sein:
\begin{itemize}
\item[(i)] $z_i$ mit $i=1,...,m$ lassen sich als Funktionen von $\v{x}, \;\v{u}$ und einer endlichen Anzahl von Zeitableitungen $u_j^{(k)}(t)$ ausdr\u cken: $\v{z}=\v{\Phi}(\v{x},\v{u},\vdiff{u},...,\v{u}^{(\alpha)})$
\item[(ii)] $\v{x}$ und $\v{u}$ lassen sich als Funktion von $z_i(t)$ und einer endlichen Anzahl von Zeitableitungen $z_i^{(k)}(t)$ darstellen: $\v{x}^*=\v{\theta}_{\v{x}}(\v{z}^*,\vdiff{z}^*, ..., \v{z}^{*(\beta-1)})$, $\v{u}^*=\v{\theta}_{\v{x}}(\v{z}^*,\vdiff{z}^*, ..., \v{z}^{*(\beta)})$ mit Vorgabe $\v{z}^*(t)$.
\item[(iii)] Die Komponenten von $z$ sind differenziell unabh\a ngig, d.h. es ex. keine DGL: $\v{\psi}(\v{z}^*,\vdiff{z}^*, ..., \v{z}^{*(\delta)})=\v{0}$. \A quivalent zu $\dim \v{z}=\dim \v{u}$.
\end{itemize}
Ausgang darstellbar \u ber $\v{y}=\v{\theta}_{\v{y}}(\v{z}^*,\vdiff{z}^*, ..., \v{z}^{*(\gamma)})$.\\
Der flache Ausgang ist nicht eindeutig. Freie Systeme $\vdiff{x}=\v{f}(\v{x})$ sind nicht flach, da (iii) nicht erf\u llt. Triviale F\a lle, wie $\v{z}=[\v{x}^T, \v{u}^T]^T$, sind ausgeschlossen durch (iii). LTI und LTV Systeme sind genau dann steuerbar, wenn sie flach sind.\\

\paragraph{Flache Eingr\o \ss ensysteme}
Mit $\dim\v{u}=1$ reduzieren sich die Bedingungen auf $z=\phi(\v{x})$, was auf $\v{x}=\v{\theta}_{\v{x}}(z,...,z^{(n-1)}), u=\theta_u(z,...,z^{(n)}), y=\theta_y(z,...,z^{(n-r)})$ f\u hrt, mit relativem Grad $r$.

\paragraph{Zusammenhang zwischen Flachheit und Exakter EZ-Lin}
Fiktive Ausgangsgr\o \ss bei Exakter E/Z-Lin stellt einen flachen Ausgang dar. Flachheit umfasst E/Z-Lin.


\subsubsection{Flachheitsbasierte Trajektorienplanung und Steuerung}
Ausgangsgr\o \ss e folgt einer vorgegebenen Referenztrajektorie. Besonders einfach f\u r flache Systeme.
\paragraph{Realisierung von \U berg\a ngen zwischen AP}
AP ist die L\o sung von $\v{0}=\v{f}(\v{x}_R,\v{u}_R)$. AP entsprechen der Form aus \textbf{(i)}. F\u r flache Systme kann die Vorgabe der AP entweder durch Vorgabe von $\v{u}_R^i$ oder von $\v{z}_R^i$, $i=0,\tau$ erfolgen. $\v{z}^*$ als hinreichend of stet.diff.bar mit Stationarit\a tsbed. bei $0$ und $\tau$ f\u r $\v{z}(\cdot)$ und deren $\beta_i$-fachen Ableitungen. F\u hrt auf (ii).

\paragraph{Konstruktion polynomialer Solltraj.}
Um $\v{z}$ zu bestimmen w\a hle Polynom vom Grad $2\beta +1$ resultiert in einem algebraischen Gleichungssystem f\u r die Koeffizienten $p_i$ des Polynoms.

\paragraph{2-FHG-Regeleung mit flachheitsbasierter Steuerung}
Praxiseinsatz. Flachheitsbasierte Parametrierung + Referenzgenerator und eine R\u ckf\u hrung mit Beobachter zur Stabilisierung des Trajektorienfolgefehlers. Bei kleinen Abweichungen zwischen $\v{x}_0$ und $\v{x}_0^*$ kann eine Linearisierung aus der flachheitsbasierten Parametrierung folgen, was auf ein LTI System f\u hrt.

\subsubsection{Flachheitsbasierte Trajektorienfolgeregelung}
Trajektorienfolgeregelung f\u r flachen Ausgang $\v{z}$. Zwei F\a lle \textbf{(i)} $\sum_{i=1}^m \beta_i=n$ und \textbf{(ii)} $\sum_{i=1}^m \beta_i>n$.

\paragraph{(i) Statische Zustandsr\u ckf\u rhung}
System besitzt vektoriellen relativen Grad $r=\{r_1,...,r_m\}=\{\beta_1,...,\beta_m\}$ und ist damit exakt E/Z-lin.bar. Mit Diffeomorphismus $\v{\xi}=\v{z}=\v{\Phi}(\v{x})$ in nichtlineare RNF transformieren. Mit neuem Eingang $\v{v}(t)$ gem\a \ss\ $\xi_i^{(\beta_i)}=v_i$ (Brunovsky-NF) kann eine exakte Linearisierung und Entkopplung erzielt werden mit $\v{u}=\v{a}^{-1}(\v{\xi},\v{v})$. Vorgabe der Eigenwerte f\u r die Integratorketten.\\
Flachheitsbasierte Trajektorienfolgeregelung ergibt sich aus Eingangsparametrierung $\v{u}=\v{\Theta}_{\v{u}}(z_1,...,z_1^{(\beta_1 -1)}, v_1, ..., z_m,...,z_m^{(\beta_m -1)}, v_m)$. Statisch da nur $v$ ohne Ableitung vorkommt.
\paragraph{(ii.a) Quasi-Statische Zustandsr\u ckf\u rhung}
Suche Diffeomorphismus der in verallgemeinerte RNF transfomiert. Definiere $\kappa$ so, dass $\sum_{i=1}^m \beta_i=n+\sum_{i=1}^m \kappa_i$. 
Neue Zust\a nde $\v{\xi}$ mit $\beta_i-\kappa_i$ beschreiben. Neue Eing\a nge als $m$ lineare Integratorketten der L\a nge $\beta_i-\kappa_i$. 
In der quasi-statischen Zustandsr\u ckf\u hrung treten nun Ableitungen $v_i^{(j)}, \; j=1,...,\kappa_i$ auf. Systemordnung $n$ bleibt erhalten in den transformierten Koordinaten $\v{\xi}$. Bei Trajektorienfolgeregelung: $z_i^*(t) \in C^{\beta_i}$.

\paragraph{(ii.b) Dynamische Zustandsr\u ckf\u rhung}
Geeignete dynamische Erweiterung, um dann wie eine statische ZuRuFu aus (i) zu entwerfen. Ansatz:
\begin{align*}
&\vdiff{x}_d=\v{f}_d(\v{x},\v{x}_d, \v{u}_d) \quad \v{u}=\v{b}(\v{x},\v{x}_d, \v{u}_d) \quad \dim\v{x}=\kappa\\
&\Rightarrow \quad \vdiff{x}=\v{f}(\v{x},\v{b}(\v{x},\v{x}_d, \v{u}_d)) \quad \vdiff{x}_d=\v{f}_d(\v{x},\v{x}_d, \v{u}_d)
\end{align*}
sodass $\dim[\v{x},\v{x}_d]=n+\kappa=\sum_{i=1}^m \beta_i$

\subsubsection{Nichtlinearer Folgebeobachter mit zeitvarianter Beobachterverst\a rkung}
Nichtlinearer Zustandsbeobachter als Ansatz: $\vhatdiff{x}=\v{f}(\hat{\v{x}}, \v{u})+L(t)(\v{y}-\v{h}(\hat{\v{x}}))$ mit Beobachterkorrektur $L(t)$. Dimensionierung von $L(t)$ aus der Beobachterfehlerdynamik $\vtildediff{x}$ im Zustand $\tilde{\v{x}}(t)=\v{x}(t)-\hat{\v{x}}(t)$. Aus der Beobachterfehlerdynamik folgt: $\vhatdiff{x}=(A(t)-L(t)C(t))\tilde{\v{x}}$ mit $A(t)=\frac{\partial \v{f}}{\partial \v{x}}\vert_{\v{x}^*,\v{u}^*}$ und $C(t)=\frac{\partial \v{h}}{\partial \v{x}}\vert_{\v{x}^*,\v{u}^*}$. Dimensionierung bspw aus Ackermann-Formel f\u r LTV-Systeme.

\subsubsection{Trajektorienfolgeregelung f\u r einen nicht-flachen Ausgang}
Bestimme Referenztrajektorie $\v{z}^*(t)$ aus DGL $\v{\theta}_{\v{y}}(\v{z}^*,\vdiff{z}^*, ..., \v{z}^{*(\gamma)})=\v{y}^*$ um eine Folgeverhalten f\u r $\v{y}(t)\rightarrow \v{y}^*(t)$ zu implizieren. DGL ist stabil: \U ber numerische Integration $\v{y}^*(t)$ ermitteln. DGL instabil: Numerisch l\o sen. Nur APwechsel: Direkt aus $\v{\theta}_{\v{y}}(\v{z}_R^*,\v{0},...,\v{0})=\v{y}_R^*$

\section{Nichtlinearer Beobachter}
\subsection{Nichtlineare Beobachtbarkeit}
\textsc{Beobachtbarkeit}\\
Anfangszust\a nde $\v{x}_{01} \neq \v{x}_{02}$ ununterscheidbar, wenn $y_i(t,\v{x}_{01}, \v{u})=y_i(t,\v{x}_{02}, \v{u})$ f\u r alle $\v{u}$ identisch sind. Beobachtbar wenn $\v{x}_{01}=\v{x}_{02}$. \vspace{0.2cm}

\textsc{Beobachtbarkeitsraum}\\
$O=\text{span}\{h_1(\v{x}), L_fh_1(\v{x}), ..., h_p(\v{x}), L_f h_p(\v{x}),... \}$ \vspace{0.2cm}

\textsc{Gleichf\o rmige Beobachtbarkeit}\\
Systeme, deren Zust\a nde aus aufeinanderfolgenden Zeitableitungen der Messgr\o \ss en rekonstriert werden k\o nnen sind gleichf\o rmig beobachtbar: $H \subset O$, mit $H$ wie Beobachtbarkeitsraum nur bis $L_f^{\rho_i-1}$ mit Beobachtbarkeitsindex $\rho_i$. \vspace{0.2cm}

\textsc{Beobachtbarkeits-Abbildung}\\
Wenn gleichf\o rmig beobachtbar, dann Beo-Abb. $\v{q}(x)$ immer invertierbar (lokal), d.h. der unbekannte Zustand $\v{x}$ kann aus dem Vektor $\bar{\v{y}}$ der Zeitableitungen eindeutig berechnet werden. $\v{q}(\v{x})=[\v{q}_1 \cdots \v{q}_p]^T = \bar{\v{y}}$ mit $\v{q}_i=[L_f^0 h_i(\v{x}) \cdots L_f^{n_i-1}h_i(\v{x})]^T$. \vspace{0.2cm}

\textsc{Lokale Beobachtbarkeit}\\
System lokal beobachtbar an $\v{x}_p$ $\Leftrightarrow$ $\exists$ Umgebung $U$ von $\v{x}_p$, so dass $\v{q}$ f\u r $\v{x}\in U$ invertiert werden kann. \vspace{0.2cm}

\textsc{Beobachtbarkeits-Matrix}\\
Jacobi-Matrix von $\v{q}(\v{x})$ hat vollen Rang an $\v{x}_p$, dann ist das System lokal beobachtbar an $\v{x}_p$.


\subsection{Beobachterentwurf f\u r nichtlineare Systeme}
\textsc{Luenberger Struktur}\\
$\vhatdiff{x}=\underbrace{f(\hat{\v{x}},\v{u})}_{\text{Simulator}}+\underbrace{G(\hat{\v{x}},\v{u})(\v{y}-\v{h}(\hat{\v{x}}))}_{\text{Korrektur}}$

\textsc{Dimensionierung}\\
Sinnvoller Anfangswert. Korrektur \u ber Beobachterfehler $\tilde{\v{x}}=\v{x}-\hat{\v{x}}$ mit Ziel $\lim_{t\rightarrow \infty}\tilde{\v{x}}(t)=\v{0}$ (Stabilisierungsaufgabe).

\subsubsection{Spezialf\a lle}
\paragraph{a) Linearisierung um Ruhelage $\v{x}_R$}
$\tilde{\v{x}}=\Delta\v{x}-\Delta\hat{\v{x}}$ f\u hrt auf $\vtildediff{x}=(A-GC)\tilde{\v{x}}+\mathcal{O}^2$ und somit beobachtbar, wenn $\text{Re}\{\lambda(A-GC)\}<0$.\\
Gegenbsp: $\dot{x}=x, \; y=x^3$ mit $x_R=0$. Linearisiert nicht beobachtbar, obwohl NLS beobachtbar. W\a hle: $\dot{\hat{x}}=\hat{x}+g\underbrace{(\sqrt[3]{y}-\hat{x})}_{\tilde{x}}$ $\Rightarrow$ $\dot{\tilde{x}}=\tilde{x}(1-g)$

\paragraph{b) Spezielle Systemstruktur}
Kein Plan, ob wichtig




